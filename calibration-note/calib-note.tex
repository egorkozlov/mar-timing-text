
\documentclass[12pt,letter]{article}
\usepackage[left=0.8in,right=0.8in,top=1in,bottom=1in]{geometry}
\usepackage{amsmath}
\usepackage{pdflscape}
\usepackage{amsfonts}
\usepackage{amssymb}
\usepackage{graphicx}
\usepackage{caption}
\usepackage{multicol}
\usepackage{microtype}
\usepackage{euscript}
\usepackage{epsfig}
\usepackage{epstopdf}
\usepackage{mathrsfs}
\usepackage{tikz}
\newcommand{\hypo}{\mathcal{H}}            
\bibliographystyle{ieeetr}

\usepackage[flushleft]{threeparttable}

%\usepackage[cp1251]{inputenc}
\usepackage[english]{babel} 
\DeclareMathOperator{\rank}{rank}
\newcommand*{\hm}[1]{#1\nobreak\discretionary{}
            {\hbox{$\mathsurround=0pt #1$}}{}}

            \def\onepc{$^{\ast\ast}$} \def\fivepc{$^{\ast}$}
\def\tenpc{$^{\dag}$}
\def\legend{\multicolumn{4}{l}{\footnotesize{Significance levels
:\hspace{1em} $\dag$ : 10\% \hspace{1em}
$\ast$ : 5\% \hspace{1em} $\ast\ast$ : 1\% \normalsize}}}


\newcommand{\bs}[1]{\boldsymbol{#1}}  
\newcommand{\bsA}{\boldsymbol{A}}

%\setstretch{1}                         
\flushbottom                            
\righthyphenmin=2                      
\pagestyle{plain}                       
%\settimeformat{hhmmsstime}  
\widowpenalty=300                   
\clubpenalty=3000                     
\setlength{\parindent}{0em}           
\setlength{\topsep}{0pt}              
\usepackage[pdftex,unicode,colorlinks=true,urlcolor=blue]{hyperref}
\usepackage{bbm}
\renewcommand{\emptyset}{\varnothing}

\setlength{\parskip}{0.5\baselineskip plus2pt minus2pt}

\newcommand{\e}{\varepsilon}
\DeclareMathOperator*{\Argmax}{\mathrm{Argmax}}
\DeclareMathOperator*{\Argmin}{\mathrm{Argmin}}
\DeclareMathOperator*{\argmax}{\mathrm{arg\,max}}
\DeclareMathOperator*{\argmin}{\mathrm{argmin}}

\newcommand{\blp}{\mathrm{BLP}}
\DeclareMathOperator*{\plim}{\mathrm{plim}}
\DeclareMathOperator{\Max}{\mathrm{Max}}
\newcommand{\R}{\mathbb{R}}
\newcommand{\Y}{\mathcal{Y}}
\newcommand{\Z}{\mathcal{Z}}
\renewcommand{\geqslant}{\geq}
\renewcommand{\leqslant}{\leq}
\newcommand{\p}{\bs p}
\newcommand{\y}{\bs y}
\def\dd#1#2{\frac{\partial#1}{\partial#2}}

\renewcommand{\emptyset}{\varnothing}


\DeclareMathOperator{\tr}{\mathrm{tr}}

\newcommand{\bb}{\bs \beta}
\newcommand{\X}{\bs X}
\DeclareMathOperator{\E}{\mathbb{E}}
\DeclareMathOperator{\PP}{\mathbb{P}}
\DeclareMathOperator{\V}{\mathbb{V}}
\DeclareMathOperator{\CM}{\mathbb{C}}
\renewcommand{\C}{\CM}
\DeclareMathOperator{\var}{\mathrm{var}}
\DeclareMathOperator{\cov}{\mathrm{cov}}
\DeclareMathOperator{\corr}{\mathrm{corr}}
\DeclareMathOperator{\MSE}{\mathrm{MSE}}
\DeclareMathOperator{\Bias}{\mathrm{Bias}}
\renewcommand{\P}{\PP}
\newcommand{\dsim}{\stackrel{d}{\sim}}
\newcommand{\hn}{\mathcal{H}_0}
\newcommand{\ha}{\mathcal{H}_a}
\newcommand{\thetab}{\bs \theta}
\newcommand{\pv}{\text{P-value}}
\newcommand{\N}{\mathcal{N}}
\newcommand{\MLE}{\scriptscriptstyle MLE}
\newcommand{\LR}{\mathrm{LR}}
\newcommand{\I}{\mathbb{I}}
\newcommand{\sumin}{\sum\limits_{i=1}^n}
\newcommand{\sumti}{\sum\limits_{t=0}^\infty}
\newcommand{\hbeta}{\hat{\beta}}
\newcommand{\halpha}{\hat{\alpha}}
\newcommand{\hsigma}{\hat{\sigma}}
\newcommand{\hvar}{\widehat{\var}}
\newcommand{\hcov}{\widehat{\cov}}
\newcommand{\Q}{\mathbb{Q}}



\newcommand{\pconv}{\xrightarrow{ \ p \ }}
\newcommand{\dconv}{\xrightarrow{ \ d \ }}
\newcommand{\asconv}{\xrightarrow{ \ a.s. \ }}
\newcommand{\msconv}{\xrightarrow{ \ m.s. \ }}

\newcommand{\pic}[4][h!]{\begin{figure}[#1]


\begin{center}\includegraphics[width=#2cm]{#3}\caption{#4\label{#3}}\end{center}
\end{figure}}

%outtex
\def\onepc{$^{\ast\ast}$} \def\fivepc{$^{\ast}$}
\def\tenpc{$^{\dag}$}
\def\legend{\multicolumn{4}{l}{\footnotesize{Significance levels
:\hspace{1em} $\dag$ : 10\% \hspace{1em}
$\ast$ : 5\% \hspace{1em} $\ast\ast$ : 1\% \normalsize}}}
%end outtex

%\bibliographystyle{ieeetr}

\newcommand{\laseq}{\stackrel{\lambda\text{-a.e.}}{=}}
\renewcommand{\d}{\underline}
\renewcommand{\u}{\overline}
\newcommand{\td}{\underline{\theta}}
\newcommand{\tu}{\overline{\theta}}
%\renewcommand{\theenumi}{\alph{enumi}}
%\renewcommand{\labelenumi}{(\theenumi)}
%\renewcommand{\theenumii}{\roman{enumii}}
%\renewcommand{\labelenumii}{\theenumii.}
%\renewcommand{\theenumiii}{\arabic{enumiii}}
%\renewcommand{\labelenumiii}{\theenumiii.}
%\renewcommand{\epsilon}{\varepsilon}
\newcommand{\hneq}{\stackrel{\hn}{=}}
\newcommand{\deq}{\stackrel{d}{=}}
%\title{416-2 Final Project\\
%Reproductive Technologies, Aging and Fertility Choice (Proposal)}
%\author{Egor Kozlov}
\begin{document}
%\begin{center}\textbf{Economics 416-1} \\ \emph{By Egor Kozlov}\end{center}
%\maketitle



%\section{Introduction}
%Not all unplanned pregnancies happen at teenage age. Among adults quite substantial share of people have kids before or at the year they marry, and this translates is associated with a great increase of share of divorced. In fact, even among college graduates 30 years old kids first females are 3.6\% as opposed to unmarried women with kids being 4.6\%. Counting only  ever married women 21--40 with kids, these women account for more than 10\% in college population and more than 25\% in general population. So the incidence of shotgun marriage is substantial.

\section{Functions}

Instantaneous utility:
\[U^{\text{no children}} = \theta^m \cdot \frac{c_m^{1-\sigma}}{1-\sigma} + \theta^f \cdot \frac{c_f^{1-\xi}}{1-\xi}.\]
\[U^{\text{with children}} = \theta^m \cdot \frac{c_m^{1-\sigma}}{1-\sigma} + \theta^f \cdot \frac{c_f^{1-\sigma}}{1-\sigma} + \phi_0 + \alpha\cdot \frac{Q^{1-\sigma}}{1-\sigma}.\]

Budget constraint (the most general) is:
\[(c_f^{1+\rho} + c_m^{1+\rho})^{\frac1{1+\rho}} + x + a' = R\cdot a + W^m_t(z_m) + l_f\cdot W^f_t(z_f).\]
\[W^i_t = \exp( z_i + \text{Trend}^i_t), \ \ \text{Trend}^i_t = \tau^i_0 + \tau^i_1 \cdot t + \tau^i_2 \cdot t^2 + \tau^i_3 \cdot t^3.\]
\[z_m' = z_m + \varepsilon^m_t, \ \ \varepsilon^m_t \sim \mathcal{N}(0,\sigma^{2}_{\varepsilon,m}), \ \ z_0 \sim \mathcal{N}(0,\sigma^{2}_{z_0,m}).\]
\[z_f' = z_f - \delta(l_f) + \varepsilon^f_t, \ \ \varepsilon^f_t \sim \mathcal{N}(0,\sigma^{2}_{\varepsilon,f}), \ \ z_0 \sim \mathcal{N}(0,\sigma^{2}_{z_0,f}).\]
\[\text{Child quality production function:} \ \ Q = \left[ x^{\lambda}  + \kappa \cdot (1-l_f)^{\lambda}\right]^{\frac1{\lambda}}.\]
\newpage
\section{Parameters}
\begin{center}
\begin{tabular}{|c|p{0.35\linewidth}|c|p{0.3\linewidth}|}\hline
 & Meaning  & Value & Identification \\\hline
\multicolumn{4}{|c|}{Utility parameters:}\\\hline
$\phi_0$ & Additive utility from having a child & 1.80 & {\footnotesize Total \% with kids} \\
$\alpha$ & Weight of child quality in utility & 0.61 & {\footnotesize steepness of age profile } \\
$\kappa$ & Relative productivity of labor & 0.84 & {\footnotesize female labor supply} \\
$\lambda$ & CES power in prod fun &  $0.7^{*}$ & {\footnotesize \textit{External:} Sommer (2016) } \\
$\xi$ & CRRA power of $Q$ in utility &  $1.5^{*}$ & {\footnotesize \textit{External:} as $\sigma$} \\
$\rho$ & RTS in partrens' consumption & $0.23^{*}$ & {\footnotesize \textit{External:} Voena (2015) } \\\hline
\multicolumn{4}{|c|}{$p^{\text{meet}}_t = p_0 + p_1\cdot (\text{Age}-30) + p_2 \cdot (\text{Age}-30)^2$}\\\hline
$p_{21}$ & Meeting probability at 21 &  $0.16$ & \\
$p_{28}$ & Meeting probability at 28 &  $0.37$ & {\footnotesize marriage rate by age} \\
$p_{35}$ & Meeting probability at 35 &  $0.39$ &  \\\hline
\multicolumn{4}{|c|}{$p^{\text{pregnant}}_t = p_0 + p_1\cdot (\text{Age}-30) + p_2\cdot (\text{Age}-30)^2$ }\\\hline
$p_{21}$ & Pregnancy probability at 21 &  $0.028$ & \\
$p_{28}$ & Pregnancy probability at 28 &  $0.031$ &  {\footnotesize shares of KF and MF couples} \\
$p_{35}$ & Pregnancy probability at 35 &  $0.009$ & \\\hline
\multicolumn{4}{|c|}{Love shock: $\psi' = \psi + \varepsilon$}\\\hline
$\sigma_\psi$ & Variance of innovation &  0.58 & {\footnotesize \% divorced by years after marriages}   \\
$\sigma_{\psi_0}$ & Relative variance of initial & $5.16\cdot \sigma_\psi$ & {\footnotesize total divorce rates} \\\hline
\multicolumn{4}{|c|}{Utility shifters:}\\\hline
$\phi_m$ & {\footnotesize Male utility lost at bargaining from if marry a single mother} &  $11.0\cdot \phi_0$ & {\footnotesize stock of divorced w and w/o kids} \\
$\phi_s$ &{\footnotesize  Utility loss at bargaining from not entering a shotgun marriage (both)} & $0.83\cdot \sigma_{\psi_0}$ & {\footnotesize KF and MF divorce difference} \\\hline
\multicolumn{4}{|c|}{${}^*$ indicates externally fixed values, all other values are SMM estimates }\\\hline
\end{tabular}
\end{center}

\section{Utility Shifters in Bargaining}
{\footnotesize
\begin{tabular}{|c||c|c||c|c||c|}\hline
 & \multicolumn{2}{|c||}{Male gets:} & \multicolumn{2}{|c||}{Female gets:} &  \\\hline
Situation & Agree & Disagree & Agree & Disagree &  NBS \\\hline
Regular match & $V^{m,c}_t(\theta)$ & $V^{m,s}_t$ & $V^{f,c}_t(\theta)$ & $V^{f,s}_t$ & $\scriptstyle \left[ V^{m,c}_t(\theta) - V^{m,s}_t\right]\times \left[ V^{f,c}_t(\theta) - V^{f,s}_t\right]$\\
Unplanned pregnancy & $V^{m,ck}_t(\theta)$ & $V^{m,s}_t - \phi_s$ & $V^{f,ck}_t(\theta)$ & $V^{f,s}_t - \phi_s$ & $\scriptstyle  \left[ V^{m,ck}_t(\theta) - V^{m,s}_t + \phi_s\right]\times \left[ V^{f,ck}_t(\theta) - V^{f,s}_t + \phi_s\right]$\\
Meeting a single mother & $V^{m,ck}_t(\theta) - \phi_m$ & $V^{m,s}_t$ & $V^{f,ck}_t(\theta)$ & $V^{f,s}_t$ & $\scriptstyle  \left[ V^{m,ck}_t(\theta) - V^{m,s}_t - \phi_r\right]\times \left[ V^{f,ck}_t(\theta) - V^{f,s}_t\right]$\\\hline
\end{tabular}
}


\newpage
\section{Moments and Fit (College Graduates)}

{
\begin{center}
\begin{tabular}{|l|l|l|}\hline
Target & Model & Data \\\hline
\multicolumn{3}{|c|}{\textbf{Hazards:}} \\\hline
Hazard of marriage by age, 22--35 & \multicolumn{2}{|c|}{see fit graphs}\\
Hazard of the first child by age, 23--35 & \multicolumn{2}{|c|}{see fit graphs}\\\hline
\multicolumn{3}{|c|}{\textbf{Transitions in couples:}} \\\hline
Divorced by years after marriage, 1--10 & \multicolumn{2}{|c|}{see fit graphs}\\
Have children by years after marriage, 1--10 & \multicolumn{2}{|c|}{see fit graphs}\\\hline
\multicolumn{3}{|c|}{\textbf{Group shares:}} \\\hline
\% Kids First in population by age, 23--35 & \multicolumn{2}{|c|}{see fit graphs}\\
\% Marriage First in population by age, 23--35 & \multicolumn{2}{|c|}{see fit graphs}\\
\% Kids First (relative), by age, 23--35 & \multicolumn{2}{|c|}{see fit graphs}\\\hline
\multicolumn{3}{|c|}{\textbf{Total stocks:}}\\\hline
Divorced at 30 if one marriage & 6.2\% & 6.9\% \\
Divorced + with kids at 30  & 2.6\% & 2.7\% \\
Never married + with kids at 30  & 2.3\% & 4.4\% \\
$>1$ marriage at 40 & 11.0\% & 12.0\% \\
\textbf{Divorced if Kids First (21--40) }&14.7\% &14.8\% \\
\textbf{Divorced if Marriage First (21--40)} &5.5\% & 5.4\% \\\hline
\multicolumn{3}{|c|}{\textbf{Household Technology:}}\\\hline
In Labor Force (30, married, kids) & 74.1\% & 74.0\% \\
Child Expenditures / Labor Earnings & 41.1\% & 40.0\% \\
\hline
\end{tabular}
\end{center}


\hspace{-1cm}\begin{tabular}{|c|c|c|}\hline
\multicolumn{3}{|c|}{\textbf{Fit Graphs:}}\\\hline
\includegraphics[scale=0.4]{haz_mar.pdf} & \includegraphics[scale=0.4]{share_kfmf.pdf} & \includegraphics[scale=0.4]{kids_yaftmar.pdf} \\\hline
\includegraphics[scale=0.4]{haz_child.pdf} & \includegraphics[scale=0.4]{share_kfrel.pdf}   & \includegraphics[scale=0.4]{div_yaftmar.pdf}  \\\hline
\end{tabular}


}


\newpage

\section{Less Nice Things (Non-Targeted Moments)}
\begin{center}
\includegraphics[scale=0.6]{men_kids_ratio.pdf} \\
\includegraphics[scale=0.6]{men_mar_ratio.pdf} \\
\includegraphics[scale=0.6]{div_kfmf.pdf} 
\end{center}


\newpage
\section{Removing The Social Stigma}
\begin{center}
\textbf{Lots of couples divorce immediately...}

\includegraphics[scale=1.0]{div_kfmf_ref.pdf} 
\end{center}


\newpage

\section{Identification Concerns}
Identification of scale of $\psi$ is not obvious.

Simple setup: 3-period model, $t \in \{0,1,2\}$ start as couple. Initial love shock is $\psi_0 \sim \mathcal{N}(0,\sigma^2_0)$, shock is $\delta \sim \mathcal{N}(0,\sigma^2_\delta)$.  Suppose $\beta = 1$ and all utility flows are fixed.

\begin{enumerate}
\item Normalize all utilities to $0$, and let males and females are perfectly symmetric. Suppose $10\%$ are divorced in period 1 and $5\%$ are divorced in period 2.

Couple divorces before entering period $2$ if $\psi_0 + \delta_1 + \delta_2 < 0$

Couple divorces before entering period $1$ if $\psi_0 + \delta_1 + \E(\psi_0 + \delta_1 + \delta_2 | \psi_0 + \delta_1 + \delta_2>0) < 0$.

Replacing $\psi_0 \to 5\cdot\psi_0$ and $\delta \to 5\cdot \delta$ do not change the inequalities, so the scale is not identifiable.

\item We need some interaction between $\psi$ and utilities to estimate size of $\psi$. Utility structure is
\[u = \theta^f\cdot \frac{c_f^{1-\sigma}}{1-\sigma} + (1-\theta^f)\cdot \frac{c_m^{1-\sigma}}{1-\sigma} + \phi_0 + \alpha \cdot \frac{\left[ x^\lambda + \kappa\cdot(1-l_f)^{\lambda}\right]^{\frac{1-\sigma}{\lambda}}}{1-\sigma} + \psi.\]

$\frac{x}{c+x + s} \approx 0.4$ identifies $\alpha$. Labor force participation identifies $\kappa$ (in fact, $\alpha$, $\kappa$. $\phi_0$ are very stable). I did quite a lot of estimations that slightly adjust the targets or add more targets without changing major things. $\sigma_{\psi_0} / \sigma_\psi$ is also always around $5$, but $\sigma_\psi$ itself is not.
\end{enumerate}

\newpage
\section{Income Process}
\begin{itemize}
\item Income data: per-hour ACS labor earnings for people who:
\begin{itemize}
\item Work full year ($\geq 48$ weeks)
\item Report ``usual hours worked'' $\geq 10$
\item Report labor income more than $\$ 3000$ per year
\item Drop top and bottom 5\% of resulted per-hour earnings
\end{itemize}
\item Trend $\tau$: cross-sectional regression of per-hour earnings on age polynomial
\item Male variances: $\sigma_{\varepsilon,m}$, $\sigma_{z_0,m}$ --- matching variance of residuals from this regression at 24 and 30. Exact match.
\item Female variances: do the same thing, get $\tilde\sigma_{\varepsilon,f}$, $\tilde\sigma_{z_0,f}$. Then set $\sigma_{z_0,f} = \tilde\sigma_{z_0,f}$ and rescale $\sigma_{z,f} = 0.9\times \sigma_{z,f}$. Result --- very close match.
\begin{center}
\begin{tabular}{|l|l|l|}\hline
\multicolumn{3}{|c|}{\textbf{Fit for the income variances}} \\
\multicolumn{3}{|c|}{ACS, college graduates, per-hour earnings} \\
\multicolumn{3}{|c|}{$W_i = \exp\{z_i + \text{Trend}_{i}\}, \ \ \ z_i' = z_i + \sigma_z\cdot \varepsilon_i$} \\\hline\hline
Target & Data & Model \\
Std of Labor Earnings at 24, male  & 0.445 & 0.420 \\
Std of Labor Earnings at 30, male & 0.452 & 0.422 \\\hline
\multicolumn{3}{|c|}{$\Rightarrow$ implied $\sigma_{z_0,m} = 0.441$, $\sigma_{z,m} = 0.033$} \\\hline\hline
Std of Labor Earnings at 24, female & 0.413 & 0.417\\
Std of Labor Earnings at 30, female & 0.425 & 0.417 \\\hline
\multicolumn{3}{|c|}{$\Rightarrow$ implied $\sigma_{z_0,f} = 0.406$, $\sigma_{z,f} = 0.9 \times 0.042$} \\\hline
\end{tabular}
\end{center}

\item \textbf{Better approach:} correct for selection.
\end{itemize}

\newpage
\section{Marriage Market}
\begin{itemize}
\item Marriage market is reduced form: every period with probability $p^{\text{meet}}_t$ single people meet single people with random but similar characteristics.
\item You only meet partners of your own age.
\item College graduates meet college graduates, high school graduates meet high school graduates.
\item Productivity $z^{\text{partner}} = \mu^i + \alpha^{i} \cdot z^{\text{own}} + \sigma^{i} \cdot \varepsilon$,  $\varepsilon \sim \mathcal{N}(0,1)$.
\item Cross-sectional regressions for females suggest $\mu^f\approx 0$, $\mu^m \approx 0.25$, $\sigma^f \approx 0.5$. Regress age-detrended per-hour wages of females of spouses for recently married couples. 
\item Assets: $\log(a^{\text{partner}}) = \mu_a + \log(a^{\text{own}} +  \sigma^{i} \cdot \varepsilon$
\item $\mu^f = 0.2$ and $\mu^m = -0.2$ \emph{so on average males have 20\% more wealth},  $\sigma^{i} = 0.1$ creates small spread around these values.
\item $\psi_0 \sim \mathcal{N}(0,\sigma^2_{\psi_0})$ is standard.
\item All random shocks are uncorrelated. Resulting wages and asset levels are correlated.
\end{itemize}

\textbf{Question:} how to simulate better approach? May use some less parametric estimates, but how to organize them... Especially assets/savings data. This does not seem crucial but I want some reasonable and simple structure.





\newpage


\section{On Dynamic Bargaining}
Imagine the bargaining problem with uncertainty ($a$, $b$ are future states)
\[V_0 = \theta^f_0u(c^f_0) + \theta^m_0 u(c^m_0) +  p_a \beta \left\{ \theta^f_{1a} u(c^f_{1a}) + \theta^m_{1a} u(c^m_{1a})\right\} +  p_b\beta \left\{ \theta^f_{1b} u(c^f_{1b}) + \theta^m_{1b} u(c^m_{1b})\right\}.\]
\[ \text{participation constraints:} \ \ u(c^f_{1a}) \geq \bar{u}^f_a, \ \ u(c^m_{1a}) \geq \bar{u}^m_a, \ \ u(c^f_{1b}) \geq \bar{u}^f_b, \ \ u(c^m_{1b}) \geq \bar{u}^m_b.\]
People rebargain after they learn if the state is $a$ or $b$.

If nothing is binding, $\theta^m$ and $\theta^f$ are unchanged, solution is $\bar{c}$ and the continuation value is
\[\bar V_0 = \theta^f_0u(\bar c^f_0) + \theta^m_0 u(\bar c^m_0) +  p_a \beta \left\{ \theta^f_{0} u(\bar c^f_{1a}) + \theta^m_{0} u(\bar c^m_{1a})\right\} +  p_b\beta \left\{ \theta^f_{0} u(\bar c^f_{1b}) + \theta^m_{1} u(\bar c^m_{1b})\right\}.\]

Suppose in state $a$ for $f$ is binding (so $\theta_a^f$ is adjusted by $\mu^f_a$) and in state $b$ constraint $m$ is binding (so $\theta_b^m$ is adjusted by $\mu^m_b$). Now
\[\tilde{V}_0 = \theta^f_0u(c^f_0) + \theta^m_0 u(c^m_0) +  p_a \beta \left\{ (\theta^f_{0} + \mu^f_a) u(c^f_{1a}) + \theta^m_{0} u(c^m_{1a})\right\} +  p_b\beta \left\{ \theta^f_{0} u(c^f_{1b}) + (\theta^m_{1}+\mu^m_b) u(c^m_{1b})\right\}.\]
So $c^i_j$ are chosen to maximize $\tilde{V}_0$.

\textit{Does this mean that binding participation constraints increase continuation value?} \textbf{No!}

$\mu$ are just Lagrange multipliers! True problem is 
\begin{align*}L_0 = \theta^f_0u(c^f_0) + \theta^m_0 u(c^m_0) +  p_a \beta \left\{ \theta^f_{0} u(c^f_{1a}) + \theta^m_{0} u(c^m_{1a})\right\} +  p_b\beta \left\{ \theta^f_{0} u(c^f_{1b}) + \theta^m_{1} u(c^m_{1b})\right\} + \\ +\beta \mu^f_a [u^f(c_{1a}) - \bar{u}^f_a] + \beta \mu^m_b [u^m(c_{1b}) - \bar{u}^m_b] .\end{align*} 

Now given a sequence of consumption $\tilde{c}$ that satisfies the constraints the value of the Lagrangian is
\[\tilde{V}_0 =  \theta^f_0u(\tilde c^f_0) + \theta^m_0 u(\tilde c^m_0) +  p_a \beta \left\{ \theta^f_{0} u(\tilde c^f_{1a}) + \theta^m_{0} u(\tilde c^m_{1a})\right\} +  p_b\beta \left\{ \theta^f_{0} u(\tilde c^f_{1b}) + \theta^m_{0} u(\tilde c^m_{0b})\right\}.\]

So even though we expect $\theta$ to change in the future the continuation values should not account for this! So all future values should be computed using the same $\theta$!

If in case $b$ the constraints cannot be satisfied, we instead get

\[\hat{V}_0 =  \theta^f_0u(\hat c^f_0) + \theta^m_0 u(\hat c^m_0) +  p_a \beta \left\{ \theta^f_{0} u(\hat c^f_{1a}) + \theta^m_{0} u(\hat c^m_{1a})\right\} +  p_b\beta \left\{ \theta^f_{0} \bar{u}^f_b + \theta^m_{0} \bar{u}^m_b \right\}.\]

So the weights are $\theta^f_0$ everywhere! It also holds that
\[\hat{V}_0 =  \theta^f_0\hat{V}^f_0 + \theta^m_0\hat{V}^m_0.\]
\end{document}
