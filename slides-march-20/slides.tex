\documentclass{beamer}
\usepackage[english]{babel}
\usepackage{amsmath,amsfonts}
\usepackage{multicol}

\usepackage{IEEEtrantools}
\usepackage{multirow}
% beamer setup

\usetheme{default}
\usecolortheme{seahorse}
%\usecolortheme{dove}

\setbeamertemplate{navigation symbols}{}


\usepackage{tikz}
\usetikzlibrary{shapes,arrows}
\usetikzlibrary{positioning}
\tikzstyle{block} = [rectangle, draw, rounded corners]
\tikzstyle{line} = [draw, -latex']


\DeclareMathOperator*{\argmin}{argmin}
\DeclareMathOperator*{\argmax}{argmax}
\DeclareMathOperator{\E}{\mathbb{E}}
\DeclareMathOperator{\I}{\mathbb{I}}

\AtBeginSection[]{
  \begin{frame}[plain]
  \addtocounter{framenumber}{-1}
  \vfill
  \centering
  %\begin{beamercolorbox}[sep=8pt,center,shadow=true,rounded=true]{title}
    %\usebeamerfont{title}
    \Huge{\insertsectionhead\par}%
  %\end{beamercolorbox}
  \vfill
  \end{frame}
}


\newcommand{\backupbegin}{
   \newcounter{framenumberappendix}
   \setcounter{framenumberappendix}{\value{framenumber}}
}
\newcommand{\backupend}{
   \addtocounter{framenumberappendix}{-\value{framenumber}}
   \addtocounter{framenumber}{\value{framenumberappendix}} 
}




\title{The Economics of Shotgun Marriage}
\subtitle{and Household Bargaining}
\author{Egor Kozlov}

\institute{
  Department of Economics\\
  Northwestern University}
  
  
\setbeamertemplate{footline}[frame number]

  
%  \usepackage{pgf}
%\logo{\pgfputat{\pgfxy(0,0)}{\pgfbox[right,base]{\footnotesize{\insertframenumber\,/\,\inserttotalframenumber}}}}
%\newcommand{\nologo}{\setbeamertemplate{logo}{}}

\begin{document}

\begin{frame}[plain]
\addtocounter{framenumber}{-1}
\date{\scriptsize}
\titlepage
\end{frame}


\begin{frame}[plain]
\frametitle{Question}
\begin{itemize}

\item Why people marry? What part of couple's value come from having children together?

\item People get pregnant and choose to marry each other. Their marriage performs worse yet they choose to enter it. Rational reasons?

\end{itemize}
\end{frame}


\begin{frame}[plain]
\frametitle{What I Do}
\begin{itemize}

\item Quantify difference of marriage performance by the timing of childbirth

\item Use a structural model matching these difference to understand sources of it

\item Quantify importance of fertility for marriage, aggregate impact of unplanned childbirths and policy counterfactuals

\end{itemize}
\end{frame}

\begin{frame}[plain]
\frametitle{Preview of Results}

\begin{enumerate}

\item Data: 
\begin{itemize}
\item Kids before marriage $\Rightarrow$ divorce more often.
\item College graduates: $\sim$ 3 times. Non-college: $\sim$ 1.2 times.
\item Robust yet not causal
\end{itemize}

\item Quantifying:
\begin{itemize}
\item Unplanned pregnancy changes outside options in family bargaining
\item Effect is present generically, its size identifies amount of distress
\end{itemize}

\item Exploring:
\begin{itemize}
\item 
\end{itemize}

\end{enumerate}
\end{frame}


\section{Identification}

\begin{frame}
\frametitle{Marriage}
Distribution of partners:
\begin{itemize}
\item $z^{\text{part}} = \mu_z +z^{\text{own}} + \sigma^{z,p} \cdot \epsilon_{z,p}$

target: regression of spouse's log-wage to own log-wage
\item $\log a^{\text{part}} = \mu_a + \log a^{\text{own}} + \sigma^{a,p} \cdot \epsilon_{a,p}$

target: $\mu_a = 0.2$ suggested by ..., small fixed $\sigma^{a,p} = 0.1$
\end{itemize}

Meeting process:
\begin{itemize}
\item $\psi_0 \sim \mathcal{N}(0,\sigma^2_{\psi,0})$. $\sigma^2_{\psi,0}$ estimated
\item $p^{\text{meet}} = g_0 + g_1\cdot \text{Age}$, $g_1<0$. $g_0$, $g_1$ estimated
\item Identifying variation: share of never married by age, share of re-married at 
\end{itemize}
\end{frame}

\begin{itemize}
\item Identified 
\end{itemize}




\begin{frame}
\frametitle{Updates Since November}
\begin{itemize}
\item Brand new codes in Python
\item Remarriage of single mothers
\item Skills depreciation because of time to childcare
\item No unplanned pregnancies in marriage: restrictive but cleaner identification
\item Two separate estimations: college grads versus non-college grads
\item Decomposition exercises as a goal
\end{itemize}
\end{frame}


\begin{frame}
\frametitle{Summary of Data Part}
\begin{itemize}
\item Kids First: those who had kids before or at the year of marriage
\item Marriage First: those who had kids at least at the next year
\item Sharp difference in divorce rates, robust and large
\item Much larger for college graduates
\end{itemize}
\begin{center}
\begin{tabular}{|c|c|c|c|}
& Kids First & Marriage First & Ratio \\
College & & & \\
High School & & & \\\hline
\end{tabular}
\end{center}
\end{frame}

\begin{frame}
\frametitle{Generic Feature?}
\begin{itemize}
\item Suppose that people:
\begin{itemize}
\item Can choose fertility 
\item But sometimgs kids arrive randomly
\end{itemize}
\item Then, negative impact of unplanned pregnancy is generic
\item If people choose not to have a kid in this period they are worse off b/c of arrival of the shock
\item How large is the effect depends on how indifferent are people between different fertility timings
\item We don't know how bad people feel
\item But this makes sense from prospective of college--non-college partition
\end{itemize}
\end{frame}



\begin{frame}
\frametitle{Generic Feature?}
\begin{itemize}
\item Bargaining model: if stress is symmetric things don't happen
\item Stress has to be asymmetric: females suffer more
\item College grads suffer more if they care more about fertility timing
\end{itemize}
\end{frame}


\begin{frame}
\frametitle{Theory}
\begin{itemize}
\item Why do people enter low quality marriage when kids happen?
\begin{itemize}
\item Women: 
\begin{itemize}
\item splitting childcare costs
\item worsening of marital prospectives otherwise
\end{itemize}
\item Men:
\begin{itemize}
\item access to their children
\end{itemize}
\end{itemize}
\item Why are college people hurt more by unplanned pregnancy?
\begin{itemize}
\item They are more sensitive to fertility timing
\end{itemize}
\end{itemize}
\end{frame}

\begin{frame}
\frametitle{Model-Free Evidence}
\begin{itemize}
\item \textit{If people are free to choose their fertility, arrival of a child unexpectedly makes them worse off}
\item Couple is together if
\[V^{fM} \geq V^{fS} \& V^{mM} \geq V^{mS}.\]
\item People worse off $\Rightarrow$ more divorce? Not necessary, as baby enters both $V^{\bullet M}$ and $V^{\bullet S}$. 
\item If values decrease asymmetrically, this can happen
\item Size of the asymmetry defines the size of the effect
\item Perfect child support enforcement = ?
\end{itemize}
\end{frame}




\begin{frame}
\frametitle{Planned Fertility}
I estimate $(\phi,\alpha,\kappa,\bar{Q})$

\[u_{\text{child}} = \phi + \alpha \cdot \frac{Q^{1-\sigma}}{1-\sigma},\]

\[Q = \left[x^{\lambda} + \kappa \cdot (1-l_f)^{\lambda}\right]^{\frac1{\lambda}}, \ \ Q\geq \bar{Q},\]

\[\text{budget constraint}: \ \ C + x = R\cdot a + W^m_t + W^f_t \cdot l_f.\]

Identification:
\begin{itemize}
\item $\phi$ --- total fertility level
\item $\alpha$ --- age structure (older $=$ more money)
\item $(\bar{Q},\kappa)$ are tricky:
\begin{itemize}
\item Share of earnings spent on $x$ 
\item \textit{Fertility by spouse's income}
\end{itemize}
\end{itemize}
\end{frame}

\begin{frame}
\frametitle{Unplanned Fertility \& Meetings}
$p^{\text{preg}} = p_0 + p_1\cdot \text{Age}$.
Identification:
\begin{itemize}
\item Share of kids first couples at 25, 30, 35 (stock): $p_0$
\item \% who married and had kids exactly at age 25, 30, 25: $p_1$
\end{itemize}
Concern: the data suggests $p_1>0$.
\end{frame}



\begin{frame}
\frametitle{Income Process}
\framesubtitle{$i \in \{m,f\}$}
\[W^i_t = \exp(z^i_t + T^i_t)\]

\begin{enumerate}
\item Trend: $ T^i_t = a^i_0 + a^i_1\cdot t +  a^i_2\cdot t^2 + a^i_2\cdot t^3.$
\begin{itemize}
\item Estimate based on cross-section per-hour wage in ACS with state and year dummies
\item Normalize wage at 25 to 1
\end{itemize}
\item Shocks: $z^i_t = z^m_{t-1} + \varepsilon^i_t$; $z^m_0 \sim \mathcal{N}(0;\sigma^2_{zi0})$; $\varepsilon^m_t \sim \mathcal{N}(0;\sigma^2_{zi})$.
\begin{itemize}
\item Pick $\sigma^2_{zi0}$ and  $\sigma^2_{zi}$ to match cross-sectional variance in log earnings at 25 and 30.
\item Outliers matter a lot: remove top and bottom $5\%$ of per-hour earnings
\item Male: $\sigma_{zm0} = 0.44$, $\sigma_{zm} = 0.033$.
\item Female: $\sigma_{zf0} = 0.41$, $\sigma_{zm} = 0.042$. \textbf{Rescale} to match the targets.
\item 3\%--4\% wage variation seems reasonable...
\item Skill depreciation
\end{itemize}
\end{enumerate}
\end{frame}

\begin{frame}
\frametitle{Labor Supply Margin}
\begin{itemize}
\item Females can reduce labor supply and use as input to child quality process
\item $Q = \left[x^{\lambda} + \kappa\cdot (1-l)^{\lambda}\right]^{\frac1\lambda}$
\item Labor supply affects current income:
\[c + x  = W^m + W^f\cdot l + R a.\]
\item Reduction in labor supply hurts female's productivity:
\[z^f_t = z^f_{t-1} + \varepsilon^f_{t} - f(l)\cdot \mu,\]
\item Currently $l$ is discrete, $l \in \{0.25,0.5,0.75,0.9\}$, corresponding $f(l) \in \{0.1,0.05,0,0\}$. $0.1$ roughly matches Adda et al (...)
\item $\kappa$ is calibrated such that average married female's labor supply is $0.75$.
\end{itemize}
\end{frame}


\begin{frame}
\frametitle{Remarriage}
\begin{itemize}
\item If woman with kids divorces, it becomes a single mother
\item Single mothers meet single males
\item If partners agree to marry they become a regular couple with a child
\item Males have a one-time utility penalty parameter for marrying a single mother. This affects bargaining:
\[NBS = \left[ V^{m,MK} - \phi_m - V^{m,S}\right] \times  \left[ V^{f,MK} - V^{f,S}\right].\]
\item Rationalizes decrease in marital prospectives for females with children
\item Estimated by matching share of remarried women with vs without children
\end{itemize}
\end{frame}

\section{Results}


\begin{frame}
\frametitle{Causal Consequences of Unplanned Pregnancy}
Graphs: divorce probabilities, labor supply / child's expenditures, earnings loss
\end{frame}

\begin{frame}
\frametitle{Decomposing College versus Non-college}
\begin{center}
\begin{tabular}{|c|c|c|c|}
& MF & KF & Ratio \\
Baseline: college & \\
+ income process of HS & \\
+ meeting probabilities of HS & \\
+ pregnancy probabilities of HS & \\
+ love shock distribution of HS & \\
+ single mother prospective of HS & \\
Comparison: high school & \\
\end{tabular}
\end{center}
\end{frame}

\begin{frame}
\frametitle{Understanding Value of Marriage}
\begin{tabular}{|c|c|c|c|c|c|}
Married by 25 & \\
Married by 30 & \\
Married by 40 & \\
Divorced by 25 &\\
Divorced by 30 &\\
DIvorced by 40 &\\
\multicolumn{4}{|c|}{Model Partss:}\\
Unplanned fertility & $+$ & $-$ & $-$ & $-$ & $-$ \\
Fertility & $+$ &  $+$ & $-$ & $-$  & $-$ \\
Love & $+$ & $+$ & $+$ & $-$ & $-$ \\
Returns to scale & $+$ & $+$ & $+$ & $+$ & $-$ \\ 
Risk-sharing & $+$ & $+$ & $+$ & $+$ & $+$ \\
\end{tabular}
\end{frame}


\begin{frame}
\frametitle{Hey}
\[\lim\limits_{n\to\infty} |f_n(\theta)| = 0 \text{a.s.}, \ \ \text{ but } \lim\limits_{n\to\infty}\sup_\theta |f_n(\theta)| \neq 0,\]

\[|Y - bX|'_b = -X\cdot 1 \cdot I(Y - bX>0) +X \cdot 1 \cdot I(Y - bX<0)  = \]
\[-X\cdot 1 \cdot I(Y - bX>0) + X\cdot 1 (1 - I(Y - bX>0))\]

\[(|Y - bX|^5)'_b = (Y-bX)^4 \cdot |Y - bX|'_b\]


\[ \E( |Y - bX|'_b ) = 0 \Rightarrow \E( -X\cdot 1 \cdot I(Y - bX>0) +X \cdot 1 \cdot I(Y - bX<0) ) = 0\]
\[ \E(X\cdot I(A) ) = \E(X|A)\cdot P(A).\]

\end{frame}

\begin{frame}
\frametitle{Consistency Theorem}
\[Q_n(\theta) \text{ (sums) } \to Q(\theta)  \text{ (expectations) }\]
show that
\[Q'(\theta) = 0 \text{ at } \theta = \theta_0\]
\end{frame}

\end{document}


