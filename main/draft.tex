
\documentclass[12pt,letter]{article}
\usepackage[left=0.8in,right=0.8in,top=1in,bottom=1in]{geometry}
\usepackage{amsmath}
\usepackage{pdflscape}
\usepackage{amsfonts}
\usepackage{amssymb}
\usepackage{graphicx}
\usepackage{caption}
\usepackage{multicol}
\usepackage{multirow}
\usepackage{microtype}
\usepackage{euscript}
\usepackage{epsfig}
\usepackage{epstopdf}
\usepackage{mathrsfs}
\usepackage{tikz}
\newcommand{\hypo}{\mathcal{H}}            
\bibliographystyle{ieeetr}

\usepackage[flushleft]{threeparttable}

%\usepackage[cp1251]{inputenc}
\usepackage[english]{babel} 
\DeclareMathOperator{\rank}{rank}
\newcommand*{\hm}[1]{#1\nobreak\discretionary{}
            {\hbox{$\mathsurround=0pt #1$}}{}}

            \def\onepc{$^{\ast\ast}$} \def\fivepc{$^{\ast}$}
\def\tenpc{$^{\dag}$}
\def\legend{\multicolumn{4}{l}{\footnotesize{Significance levels
:\hspace{1em} $\dag$ : 10\% \hspace{1em}
$\ast$ : 5\% \hspace{1em} $\ast\ast$ : 1\% \normalsize}}}


\newcommand{\bs}[1]{\boldsymbol{#1}}  
\newcommand{\bsA}{\boldsymbol{A}}

%\setstretch{1}                         
\flushbottom                            
\righthyphenmin=2                      
\pagestyle{plain}                       
%\settimeformat{hhmmsstime}  
\widowpenalty=300                   
\clubpenalty=3000                     
\setlength{\parindent}{0em}           
\setlength{\topsep}{0pt}              
\usepackage[pdftex,unicode,colorlinks=true,urlcolor=blue]{hyperref}
\usepackage{bbm}
\usepackage{tabularx}
\renewcommand{\emptyset}{\varnothing}

\setlength{\parskip}{0.5\baselineskip plus2pt minus2pt}

\newcommand{\e}{\varepsilon}
\DeclareMathOperator*{\Argmax}{\mathrm{Argmax}}
\DeclareMathOperator*{\Argmin}{\mathrm{Argmin}}
\DeclareMathOperator*{\argmax}{\mathrm{arg\,max}}
\DeclareMathOperator*{\argmin}{\mathrm{argmin}}

\newcommand{\blp}{\mathrm{BLP}}
\DeclareMathOperator*{\plim}{\mathrm{plim}}
\DeclareMathOperator{\Max}{\mathrm{Max}}
\newcommand{\R}{\mathbb{R}}
\newcommand{\Y}{\mathcal{Y}}
\newcommand{\Z}{\mathcal{Z}}
\renewcommand{\geqslant}{\geq}
\renewcommand{\leqslant}{\leq}
\newcommand{\p}{\bs p}
\newcommand{\y}{\bs y}
\def\dd#1#2{\frac{\partial#1}{\partial#2}}

\renewcommand{\emptyset}{\varnothing}


\DeclareMathOperator{\tr}{\mathrm{tr}}

\newcommand{\bb}{\bs \beta}
\newcommand{\X}{\bs X}
\DeclareMathOperator{\E}{\mathbb{E}}
\DeclareMathOperator{\PP}{\mathbb{P}}
\DeclareMathOperator{\V}{\mathbb{V}}
\DeclareMathOperator{\CM}{\mathbb{C}}
\renewcommand{\C}{\CM}
\DeclareMathOperator{\var}{\mathrm{var}}
\DeclareMathOperator{\cov}{\mathrm{cov}}
\DeclareMathOperator{\corr}{\mathrm{corr}}
\DeclareMathOperator{\MSE}{\mathrm{MSE}}
\DeclareMathOperator{\Bias}{\mathrm{Bias}}
\renewcommand{\P}{\PP}
\newcommand{\dsim}{\stackrel{d}{\sim}}
\newcommand{\hn}{\mathcal{H}_0}
\newcommand{\ha}{\mathcal{H}_a}
\newcommand{\thetab}{\bs \theta}
\newcommand{\pv}{\text{P-value}}
\newcommand{\N}{\mathcal{N}}
\newcommand{\MLE}{\scriptscriptstyle MLE}
\newcommand{\LR}{\mathrm{LR}}
\newcommand{\I}{\mathbb{I}}
\newcommand{\sumin}{\sum\limits_{i=1}^n}
\newcommand{\sumti}{\sum\limits_{t=0}^\infty}
\newcommand{\hbeta}{\hat{\beta}}
\newcommand{\halpha}{\hat{\alpha}}
\newcommand{\hsigma}{\hat{\sigma}}
\newcommand{\hvar}{\widehat{\var}}
\newcommand{\hcov}{\widehat{\cov}}
\newcommand{\Q}{\mathbb{Q}}



\newcommand{\pconv}{\xrightarrow{ \ p \ }}
\newcommand{\dconv}{\xrightarrow{ \ d \ }}
\newcommand{\asconv}{\xrightarrow{ \ a.s. \ }}
\newcommand{\msconv}{\xrightarrow{ \ m.s. \ }}

\newcommand{\pic}[4][h!]{\begin{figure}[#1]


\begin{center}\includegraphics[width=#2cm]{#3}\caption{#4\label{#3}}\end{center}
\end{figure}}

%outtex
\def\onepc{$^{\ast\ast}$} \def\fivepc{$^{\ast}$}
\def\tenpc{$^{\dag}$}
\def\legend{\multicolumn{4}{l}{\footnotesize{Significance levels
:\hspace{1em} $\dag$ : 10\% \hspace{1em}
$\ast$ : 5\% \hspace{1em} $\ast\ast$ : 1\% \normalsize}}}
%end outtex

%\bibliographystyle{ieeetr}

\newcommand{\laseq}{\stackrel{\lambda\text{-a.e.}}{=}}
\renewcommand{\d}{\underline}
\renewcommand{\u}{\overline}
\newcommand{\td}{\underline{\theta}}
\newcommand{\tu}{\overline{\theta}}
%\renewcommand{\theenumi}{\alph{enumi}}
%\renewcommand{\labelenumi}{(\theenumi)}
%\renewcommand{\theenumii}{\roman{enumii}}
%\renewcommand{\labelenumii}{\theenumii.}
%\renewcommand{\theenumiii}{\arabic{enumiii}}
%\renewcommand{\labelenumiii}{\theenumiii.}
%\renewcommand{\epsilon}{\varepsilon}
\newcommand{\hneq}{\stackrel{\hn}{=}}
\newcommand{\deq}{\stackrel{d}{=}}
\title{Kids First or Marriage First?}
\author{Egor Kozlov}
\begin{document}
%\begin{center}\textbf{Economics 416-1} \\ \emph{By Egor Kozlov}\end{center}
\maketitle



\begin{abstract}
Unplanned pregnancies often trigger marriage for unmarried cohabitating couples. I show that on average marriages in which a child appeared before or at the year of marriage are characterized with higher divorce probabilities and other outcomes suggesting lower marital quality. This association is robust to different ways of controlling for composition, and it is more pronounced for higher income and more educated couples. I argue that this pattern is consistent with household bargaining models, and the main driver of the result is unequal exposure of spouses to the childcare costs: since being single mother is a bad outcome for an educated female, she is more likely to agree to enter the marriages of lower quality. I estimate a lifecycle model with dynamic bargaining and fertility decisions to show how much of these differences is driven by household bargaining and how unplanned pregnancies creating unexpected childcare costs contribute to observable divorce rates.
\end{abstract}

\section{Introduction}
As of 2018, [40\%] of new births occurred by unmarried mothers. This number is rising for decades, but, more importantly, it is rather heterogeneous, reaching three-quarters for mothers who are African American, [...] under 25 and [...] for high school or less in education. But even for educated and older women this share is substantial [...]. An increase in non-marital cohabitation is also a documented trend, despite of this, persistent non-marital cohabitation is still not the most common thing: according to National Survey of Family And Growth, median duration of cohabitation is 22 months, and median duration of cohabitation if it ends by marriage is 21.\footnote{Source: Copen et al, \cite{copen}}. Unmarried unions with children are still pretty rare: according to Payne, 2013\nocite{payne}, only about 3\% of children live in unions with unmarried cohabitating parents (as opposed to 21\% with single mothers). 

These numbers have important economic implications: children born by unmarried women, as documented by, for instance Kearney, Levine, 2017, display significantly worse future outcomes in terms of education and income, and this pattern survives controlling in many dimensions. Although it is well understood that children of never married single mothers or children of blended families (when a woman marries someone else than child's father) have generally worse outcomes, even when parents marry each other after they made a baby their marriage is statistically less stable, as I argue in this paper, and the larger chances of marriage dissolution do affect future prospects of the children. 

This pairs well with common perception that some couples marry at or shortly after the moment they make a baby. If cohabitation is viewed as uncommitted relationship, unexpected fertility may serve as a trigger to obtain more commitment. Narrowly, we can interpret this as shotgun marriage story, however in the times with large abortion access many unplanned pregnancies are aborted, and cohabitating couples seem to \emph{choose} to keep an unplanned baby. The whole role of marriage changes as it is more socially acceptable to cohabitate, though it is much more conventional to do it before having kids. As highlighted by, for instance, Lafortune, 2018, despite of general decrease of role of marriage in the society, securing investment to children is one of the most important forces that makes people marry each other.

This paper highlights one important aspect of this: I argue that when marriage decision is driven by having an unexpected child, bargaining friction makes marital outcome worse. The source of this friction is the large childcare costs and unequal relative exposure to them: as women bear relatively more childcare costs, they are more likely to enter relatively lower quality marriages as a commitment to share a part of these costs. This can rationalize observable differences in marital outcomes between couples with and without out-of-wedlock births. 

Namely, I document that females who had their first birth before their first marriage have higher chances of divorce in the future, this is mainly driven by women married to the parents of their children, and this is robust and relatively more pronounced for high education or high income groups. I then provide a model incorporating the bargaining mechanism that can rationalize these differences. I estimate the model parameters to match observable patterns in the data and show what it implies about the role of unequal exposure to childcare costs in observable marriage selection and divorce patterns. Shortly, the model suggests if people could perfectly control timing of their births, general share of divorced people would decrease by [...], especially in high-income groups.

Section 2 of this paper discuss existing results and general trends that are already described by the literature. Section 3 discuss how I construct the dataset and timing variables, and Section 5 establishes main observations about differences in groups depending on their marriage timing, together with a number of additional considerations useful to understand the content of these differences. After that I explain theoretical model in Section 5 and the process of setting its parameters in Section 6. Section 7 describes estimation results and counterfactual simulations implied by these results. Section 8 provides more general discussion and concludes.

\section{Related Literature}

My work builds on few branches of literature, following long tradition of modeling fertility and marriage markets in family economics. I am not aware of any studies that cover precisely the narrow issue I focus on, but there are few recently established result that cover related aspects and compliment what I do. Three main building blocks that I utilize are literature on marriage selection and dynamic bargaining, few detailed lifecycle models of fertility and unplanned pregnancies and number of descriptive studies of microdata describing reduced form effects of out-of-wedlock births, trends in divorce rates and heterogeneity in time use of parents.

The main conceptual framework that I utilize are models of household's dynamic bargaining and limited commitment, extensively reviewed by Chiaporri, Mazzocco 2015.\nocite{chiappori-review} These models were used to answer various marriage and divorce related questions. Theoretical part of my model is mostly based on Voena, 2015\nocite{voena-1}, that uses dynamic bargaining to consider impact of changes in divorce laws, and Low et al, 2018\nocite{low-1}, focusing on marriage and divorce selection related to welfare utilitzation. There are few more recent papers utilitizing similar approaches, for instance, Shephard, 2019\nocite{shephard} studying gender and age pay gaps through equilibrium on marriage markets. 

To model fertility choices and shocks, I rely on recent lifecycle models. I make many choices similar to Sommer, 2015\nocite{sommer} who studies how changes in reproductive horizon affect couple's fertility decisions with in the environment with income shocks, paying detailed attention to time and money structure of childcare. More detailed treatment of unintended pregnancies is done by Ejrnæs, Jørgensen, 2018\nocite{ejrnaes}, they extend the model to allow for non-random abortion decisions though do not model childcare costs in details. 

Commitment in marriage and its interaction with fertility decisions attract recent attention as possible drivers of trends in marriage and divorce. Work by Lafortune, Low 2018\nocite{lafortune-1} uses ACS data and housing prices, as well as marriage market model, to argue that married people buy houses as a device to commit not to divorce, that allows them to secure investment in their future children. This is parallel and complimentary with my results, that mainly focus on consequences of not having enough commitment and how they relate to fertility timing. 

Impact of non-marital childbearing and parents divorce are well-documented empirically from children prospective. For example, Kearney, Levine 2017 use PSID to document how being born out of marriage statistically child's outcome in school years and in early adult life, and the heterogeneity of this by parent's education and age groups. Shortly, they find presence of ``marriage premium'' for children at each education level, though for less educated and younger mothers benefits of having married parents are more pronounced in early life, and for elder and more educated mothers --- in adult life. A classic study by Akerlof, 1996\nocite{akerlof} highlight the mechanism why unplanned pregnancies may not be aborted in couples so they are still present even with large abortion access. Finally, Ginther, Pollak, 2004 show that outcomes of children in blended families (i.e. having step-parents) are similar to outcomes of children of single parents. 

This paper fills an important gap in the literature considering impact of out-of-marriage births for women. As fertility, marriage and divorce involve many individual decisions affected by many unobservable factors, it is not possible to ask casual questions by using only the data, and existing empirical studies only provide descriptive evidence. This study contributes to the growing literature of quantitative models that both use the tools of family economics to endogenize timing of fertility, marriage, divorce and their interactions, and, on the other hand, provide numbers and patterns consistent with microdata, so they can be used to answer policy questions.

As noted by Ginther and Pollak, causal questions in the context of marriage and divorce should be taken with caution: to answer, for instance, how parents' divorce affect child's future we have to be precise what counterfactual we compare it with (either parents never divorced, parents stayed together for few more years, parents never married). The same issues concern impact of unplanned pregnancies in cohabitating couples, therefore, instead of trying to ask causal quesitons I the main goal of using the model is to deliver counterfactual outcomes, that still facilitate understanding of some causality.

\section{Empirical Strategy}
The key empirical observation is association between marriage timing and divorce probabilities. The goal is to argue that statistically women who have their kids before they are married have significantly higher chance to be divorced. American Community Survey (ACS) data are suitable for establishing this fact only of particular subsample of women: most importantly, those who had no more than one marriage in their life. To re-confirm the patterns I additionally use Survey of Income and Program Participation (SIPP) data: they have smaller sample size but collect exact questions I care about.

\subsection{Defining Variables}

General strategy involves dividing sample in two groups: couples who had kids before their marriage ($K\to M$) and those who had kids after their marriage ($M\to K$). After this division I perform multiple comparisons of these groups: the main comparison is looking at divorce rates in several variations, few additional comparisons involve marriage inequality and outcomes of children.

Although the couples are the primary focus of the paper, the surveys I use look at individuals and households. Therefore I follow females rather than males: upon divorce, children are more likely to stay with mothers (share of single mothers in ACS is ..., single fathers ...). The timing I consider in the  most general form is (for females, given both events occurred): 
\[\Delta T = T^{\text{First child  is born}} - T^{\text{First marriage happened}},\]
where $T$ refers to moments of time. After that I define two groups (given $\Delta T$ is measured in years):
\begin{align*}&M\to K \Leftrightarrow \Delta T \in \{1,2,...\},\\
&K\to M \Leftrightarrow \Delta T \in \{...,-2,-1,0\},\end{align*}

In practice, I exclude observations with $\Delta T < -5$ or $\Delta T > 10$. The reason for the first one is mainly related to step-children: for the subsample when it can be tracked, share of step-children is becomes higher with more negative $\Delta  T$ in ACS, and having step-children that are relatively adult at the moment of marriage is a complex issue that is beyond the scope of the paper. See a note ... in the appendix about this. The reason for the second one is mainly data quality (...).

As the evidence is mainly cross-sectional, I refer to share of divorced people in a group as divorce rates. Unlike fertility, people can change their marriage status all the time, so there is no such thing as completed divorce rate, therefore I just use the share of people who ever had a divorce. More accurate measuring of divorce rates involves conditioning on age, however fixing age restricts the attention to a particular cohort, and pooling people of different ages together allows to think about more general population.


\subsection{Sample Restrictions}
\subsubsection{ACS data}
I take ACS samples of 2008--2016, as only they have variables concerning the duration of current marital status. Namely, people report the year of their most recent marriage. Fertility history is not observed perfectly: the survey reports age of the eldest child residing with individual. However, both of these variables can recover the timing of interest for relatively large subpopulation of females who:
\begin{enumerate}
\item Had exactly one marriage, so they are either married or divorced now (treating legally separated, spouse absent and widowed as married)
\item Have children present in household, with age of the eldest child below 18
\item Most likely reside with their eldest child judging by mother's age (I pick 21--40)
\end{enumerate}
%The last restriction can be relaxed, but it allows to track partners and children more clearly and also insures better quality of self-reported data.

The most important restriction here is being married once --- that means that people who re-marry drop out of sample. For instance, if it take longer to re-marry for the people who had kids before their first marriage, share of divorced people in the group $K\to M$ would be larger without differences in probability of their first marriage to dissolve. This concern can only partially be addressed with ACS data, but this is not an issue in SIPP. 


All mean and quantile calculations are performed using person weights.

\subsubsection{SIPP data}

I use wave 1 of 2014 SIPP. It it representative for the whole population, I use person weights in all calculations.

SIPP has much more detailed information on marital history and fertility: it explicitly asks about times of first marriage and first births, it also asks about multi-partner fertility. However it has substantially smaller sample size. For SIPP I relax the restrictions and consider all females who had at least one marriage and are of age 20--40 (treating married with spouses absent, separated and widowed as married), as their marital and fertility history is recovered perfectly.

%When dividing people by income, however, I have to further restrict attention to households that consist of mother + spouse/partner (if reported) + children. This creates some attrition especially for young people, but differences in defining household members are minimal between ACS and SIPP, so it is not likely that it creates an issue.

\subsection{Defining Income}
This part describes how I define income partitions in ACS and SIPP. Shortly, I define income groups based on per hour income of working females, by excluding those who does not work from income considerations. In the appendix \ref{inc-part} I try several alternative partitions, shortly they do not affect the conclusions although they do affect magnitudes in different groups. 

To define per-hour income, I restrict attention to working women only. I define working as those who are employed now, report to work more than 25 weeks in the last year, report non-missing labor income above or equal to $\$1000$ per year, report non-missing usual working hours above or equal to $5$ hours per week.

As weeks worked last year are defined only on intervals for ACS, I impute them using the midpoints of respective intervals, then define yearly hours as a product of usual hours and imputed weeks worked. This allows to define log per-hour income as log-difference of yearly log wage and salary income with yearly log hours. I residualize these income data by regressing it on 4th degree polynomial in age, dummies and linear trends in age for each US state and dummies for each ACS survey year.

In SIPP 2014, I perform the similar operations. I exclude females who had less than $1000/12$ monthly income or had average weekly hours less than 5 during the last month. Then I divide monthly income on four times average weekly hours to get estimate for per-hour earnings. As before, I residualize logs of these income by regressing them on the same factors, except that instead of year controls (I only use 2014) I use month controls that allow to capture variation in number of work days per month. 

\section{Reduced Form Evidence}
\subsection{Raw Comparisons}
First, I compare divorce probabilities in cross-sections of people between $K\to M$ and $M\to K$ groups. See Table \ref{diff-raw} for this.

\begin{table}[h!]
\begin{center}
\begin{tabular}{|l|c||c|c||c||c|c|}\cline{2-7}
\multicolumn{1}{c|}{} & \multicolumn{3}{c||}{\small ACS: sample of married once} & \multicolumn{3}{c|}{\small SIPP: sample of  ever married} \\\cline{2-7}
\multicolumn{1}{c|}{}  & \multirow{2}{*}{\small \% K$\to$M}& \multicolumn{2}{c||}{\textbf{\% divorced (now)}} & \multirow{2}{*}{\footnotesize \% K$\to$M }& \multicolumn{2}{c|}{\textbf{\% divorced (ever)}} \\\cline{3-4}\cline{6-7}
\multicolumn{1}{c|}{}  &  &\small if M$\to$K &  \small if K$\to$M & & \small if M$\to$K & \small if K$\to$M \\\hline
\textit{All females, 21--40}  & $23.2\%$ & $10.1\%$ & $18.0\%$ & $22.8\%$ & $24.1\%$ & $34.8\%$ \\
\textit{All females, 35--40} & $18.0\%$ & $11.3\%$ & $23.0\%$ & $18.9\%$ & $28.1\%$ & $45.5\%$ \\\hline\hline
\multicolumn{7}{|p{0.9\linewidth}|}{\footnotesize \textbf{Education partition.}}\\\hline
\textit{High school only, 21--40} & $34.2\%$ & $13.9\%$ & $17.2\%$ & $33.4\%$ & $29.2\%$  & $34.2\%$ \\
\textit{College graduates, 21--40} & $9.9\%$ & $5.3\%$ & $14.7\%$ & $10.5\%$ & $12.9\%$ & $27.5\%$ \\\hline\hline
\multicolumn{7}{|p{0.9\linewidth}|}{\footnotesize \textbf{Income partition:} per-hour income for subsample of working women.}\\\hline
\textit{Low 30\% income, 21--40} & $30.8\%$ & $17.6\%$ & $24.1\%$ & $34.6\%$ & $30.5\%$ &  $42.5\%$ \\
\textit{Mid 40\% income, 21--40} & $25.5\%$ & $14.2\%$ & $21.5\%$ & $24.1\%$ & $28.3\%$ & $45.8\%$ \\
\textit{Top 30\% income, 21--40} & $14.1\%$ & $7.7\%$ & $18.1\%$ & $12.4\%$ & $22.9\%$  & $32.8\%$ \\\hline
\multicolumn{7}{|p{0.9\linewidth}|}{\footnotesize $\Delta T = \text{Year}^{\text{first child}} - \text{Year}^{\text{first marriage}}$. Restriction: $\Delta T \in \{-5,...,10\}$.}\\
\multicolumn{7}{|p{0.9\linewidth}|}{\footnotesize Definition: K$\to$M if $\Delta T \in \{-5,...,0\}$. M$\to$K if $\Delta T \in \{1,...,10\}$.}\\
\hline
\end{tabular}
\caption{Differences in raw means, ACS and SIPP\label{diff-raw}}
\end{center}
\end{table}

Three key observations from the table are:
\begin{enumerate}
\item About one-fifth of females have their first kid before or at the year they marry, this share quickly decreases with income and education
\item Share of divorced people among those who had their first kid before on at the year of their first marriage is larger.
\item The share of divorced people is about 1.5 higher for low income/education females and more than twice higher for high income/education.
\item The ACS result is not likely to be driven by picking only people with one marriage, as the pattern is exactly the same if I use SIPP and count for remarriages
\end{enumerate}

This alone is not enough to draw any causal conclusions: composition of these groups is drastically different. For instance, as suggested by the table, people with lower income are more likely to get divorced, so even conditional on income/education group K$\to$M couples may be poorer with each subgroup. Additionally, many other factors may drive the differences, including different ages of children (people in K$\to$M group by construction have older babies after any given number of years after marriage), regional and racial differences in attitudes to divorce, age composition, possible presence of step-children in K$\to$M couples and so on. I attempt to control for few of these things and argue that they explain these differences only partially, that suggests the presence of causal mechanism that I describe.

\subsubsection{Duration Considerations}
One obvious concern about comparing two groups is the temporal nature of the data: the variable I use captures fact that divorce happened in the past, and the exact timing may be different between the two groups. Due to cumulative nature of the divorce indicator, females who had marriage long time ago are more likely to be divorced even if their divorce chances in any given period are the same. Therefore, I it is useful to look at divorced females conditional on number of years passed from marriage, see the left panel of Figure \ref{pic-aftmar} for this.
\begin{figure}[h!]
\includegraphics[width=0.45\linewidth]{div_by_aftmar.eps}
\includegraphics[width=0.45\linewidth]{aftmar_by_mk.eps}
\caption{Divorced by years after marriage and distribution of years after marriage\label{pic-aftmar}}
\end{figure}

First, the pattern about different divorce probabilities hold almost uniformly by the ``age of marriage''. Group that has kids first both have higher share of divorced and faster growth of it as marriage progresses. Combining these two graphs allows to see that M$\to$K group are on average married for longer time at the moment of survey, but overall composition is not dramatically different. That would mean that comparing raw shares of divorced actually \emph{underestimates} the difference relative to controlling for number of years after marriage. This is confirmed by regression results that I skip, as in the following section I present the version where I control for years after marriage and few other factors.

\subsection{Adjusting for Composition}

This part documents large differences in variables that may affect divorce rates between the two groups. After that I try to flexibly capture these factors to see what part of the difference is attributed to the fact of being in K$\to$M group. These exercises use ACS data as they have more observations so the tests are less noisy. Shortly, after adding many controls, observable difference decreases about twice, but still remains sizable and significant. 


\subsection{Differences in Observables}
This section captures demographic and economic differences of K$\to$M and M$\to$K groups. Many factors could compute to the observable differences, here I quantify them and this justifies use of controls in the next part. I present these differences for ACS, SIPP tables are available in [Appendix ...]. For ACS, since M$\to$K and K$\to$M are defined only for subsample of people who were married once and have children present, I also compare this subsample with more general female population. See Table \ref{diff-contr} for the results.

\begin{table}[h!]
\begin{center}\small
\begin{tabular}{|l|c|c|c|c|}\cline{2-5}
\multicolumn{1}{c|}{} & \multirow{2}{*}{All population} & \multicolumn{3}{c|}{Married once + have kids}\\\cline{3-5}
\multicolumn{1}{c|}{}  & & Everyone & K$\to$M & M$\to$K \\\cline{2-5}\hline
\textit{Age} & 30.2    &  33.1     &       31.7     &       33.6 \\
\textit{Age at first marriage} & 24.2         &   23.3        &    23.6        &    23.2 \\
\textit{Age at first birth}  & 24.0       &     25.4     &       21.7       &     26.5 \\
\textit{Husband's age} & 34.9        &    35.8      &      34.4   &         36.2  \\
\textit{Age difference (W$-$H)} &  2.9       &      2.8      &       3.0      &       2.7  \\\hline\hline
\textit{Eldest child's age} & 8.6         &    7.8    &        10.0     &        7.1 \\
\textit{Number of children} & 1.1       &      2.0      &       2.3     &       2.0 \\\hline\hline
\textit{\% college or + (own)} & 32.1\%   &         39.5\%     &       16.9\%     &       46.3\% \\
\textit{\% college or + (husband)} & 33.3\%     &       36.4\%     &       14.2\%       &     42.7\% \\\hline\hline
\textit{\% In Labor Force} & 75.6\%    &        68.6\%    &        70.0\%        &    68.2\% \\
\textit{Own income (thousands)} & 32.2 &         38.4 &          29.1 &         41.1 \\
\textit{Hours worked} & 36.6       &     36.5    &        36.8        &    36.4\\\hline\hline
\textit{\% white (incl hispanic)} & 69.2\%      &      76.5\%      &      72.2\%      &      77.8\% \\
\textit{\% black} & 14.8\% &        6.7\% &            12.2\%       &       5.0\% \\\hline
\multicolumn{5}{|p{0.7\linewidth}|}{\footnotesize ACS, women, 20--40. Means. Everything is subsamples when applicable, income and hours are for subsample of women who are employed.}\\\hline
\end{tabular}
\caption{Differences in composition, ACS\label{diff-comp}}
\end{center}
\end{table}

Shortly, differences in composition are pretty drastic for income and education dimension (as suggested by different shares of K$\to$M in subgroups described above). Most of these things are intuitive, and this suggests importance of flexibly controlling for composition. Similar patterns are observed if I compare composition within each subgroup that I discuss above, so partitions should matter. 

\subsubsection{Statistical Tests and Controls}

In this section I formally test differences in shares for different subgroups and attempt to control for composition using regressions with many fixed effects.

In general, I perform the following exercise: I regress indicator of female being divorced on set of controls $X$ and dummy variable $\I(\text{K$\to$M})_i$. The specification I use is
\[D_i = \Delta \cdot \I(\text{K$\to$M})_i + X_i\gamma + \varepsilon_i.\]
Replacing $X_i$ with a constant corresponds to raw mean comparison described above. 

I report three possible estimates of $\Delta$:
\begin{itemize}
\item \textbf{Raw}, corresponding to differences in raw means in the previous part
\item \textbf{Controls}, where the regression includes dummies for ACS year, the US state, number of children, number of years after marriage, age of the eldest child, race and education group + 4th degree polynomial in woman's age
\item \textbf{Controls+income}, where the regression includes all of the above + 3rd degree polynomial in log female per-hour income (residualized as described above) for subsample of working women. This subsample is smaller, but has arguably better data quality.
\end{itemize}

%The most important controls are age of the eldest child and number of years after marriage. Basically, being in K$\to$M group is defined by their interaction 

See Table \ref{diff-contr} for the results. Generally, controlling for composition decreases observable difference, but this decrease is less than half in magnitude: even controlling for many observable factors, many of which may itself be results of unplanned fertility, we cannot assign all observable difference to composition of the groups. This also implicitly suggests that the income partition I choose makes sense: conditional on income group, additional controlling for income does not change the observable coefficient much. 

\begin{table}
\begin{center}
\begin{tabular}{|l|c||c|c|c|c|c|c|}\cline{2-8}
\multicolumn{1}{c|}{} & \multicolumn{7}{c|}{ACS: sample of married once}\\\cline{2-8}
\multicolumn{1}{c|}{} & \small Share of &\multicolumn{2}{c|}{\small Raw difference} & \multicolumn{2}{c|}{\small with controls} & \multicolumn{2}{c|}{\small controls + income} \\ \cline{3-8}
\multicolumn{1}{c|}{} & \small divorced & $\Delta$ & (s.e.) & $\Delta$ & (s.e.) & $\Delta$ & (s.e.) \\  \hline
\textit{All females, 20--50} & 14.3\% &  7.9\% &   (0.1\%) &    5.7\%  &    (0.2\%) &     4.6\%   &   0.2\% \\
\textit{All females, 35--40} &  17.3\% &  11.7\%  &  (0.2\%) &   6.7\% &  (0.3\%) &     6.2\%  &   (0.4\%) \\\hline\hline
\multicolumn{8}{|p{0.9\linewidth}|}{\footnotesize \textbf{Education partition.}}\\\hline
\textit{High school only, 20--50} & 18.1\% &   3.3\%  &    (0.2\%) &   8.4\% &   (0.3\%)   &  7.8\%   &  (0.5\%) \\
\textit{College graduates, 20--50} & 8.5\% &   9.4\% &   (0.2\%) &   3.9\% &   (0.3\%) &   3.2\%   &  (0.4\%) \\\hline\hline
\multicolumn{8}{|p{0.9\linewidth}|}{\footnotesize \textbf{Income partition:} working women only.}\\\hline
\textit{Low 30\% income, 20--40} &  21.6\% &  6.5\% &    (0.3\%) &    7.8\% &  (0.5\%) &   7.8\%   &  (0.5\%) \\
\textit{Mid 40\% income, 20--40} &  17.1\% &  7.3\% &   (0.3\%) &    3.6\%   & (0.4\%) &   3.6\% &    (0.4\%) \\
\textit{Top 30\% income, 20--40} & 11.1\% &  10.4\% &  (0.3\%) &    3.5\% &   (0.4\%) & 3.5\% &  (0.4\%) \\
\hline
\end{tabular}
\caption{Differences, controlled for composition, ACS\label{diff-contr}}
\end{center}
\end{table}
\subsection{Additional Comparisons}

Here I show few additional results. First, I present several graphs that suggest relation of divorce chances to child's age. Second, I use alternative variable and consider only people who divorced in the year of survey, as for them duration of marriage and age of the first child is (almost) perfectly observed. The main point of these calculations is to show that divorces seem to be more concentrated around the moment where children get older.

\subsubsection{Effects by Child's Age}
First, I show how share of K$\to$M people varies with age of the eldest child. I focus on subsample children are 18 or younger, as very few people have children above 18 living in the household. This exercise highlights a nuance in the construction of the groups: almost no one in K$\to$M group has children of age less than a year, because that would mean the same year of marriage, first birth and ACS survey. Additionally, since the population is trimmed at the age of 40, having kids around age of 18 means fairly low age of the first birth, that is typically associated with being in M$\to$K group. Left panel of Figure \ref{pics-raw} clearly depicts both issues, and this suggests the importance of using controls. In the specifications above I control both for number of years after marriage and the age of the eldest child, so this should help mitigating these issues.

Right panel of Figure \ref{pics-raw} shows evolution of divorce probabilities for the two groups. The increasing pattern is a result of cumulative nature of processes: people with elder children have more periods to get divorced. But the difference between two probabilities is illustrative: between ages 5 and 10 size of K$\to$M group changes a little, but difference becomes very pronounced.

\begin{figure}[h!]
\includegraphics[width=0.45\linewidth]{km_by_age.eps}
\includegraphics[width=0.45\linewidth]{div_by_age_raw.eps}
\caption{Share of K$\to$M couples and share of divorced, no controls\label{pics-raw}}
\end{figure}

I focus on studying difference more carefully then: first, I simply run a regression
\[D_i = \sum_{a=0}^{18} \beta_a \cdot \I(\text{Age of eldest child}_i = a) +  \sum_{a=0}^{18} \Delta_a \cdot \I(\text{Age of eldest child} = a)\cdot \I(\text{K$\to$M})_i + \varepsilon_i,\]
and plot $\Delta_a$.

Second, for a subsample of working women I run a regression with controls that were used for Table \ref{diff-contr} above (including polynomials in income),
\[D_i = \sum_{a=0}^{18} \Delta_a \cdot \I(\text{Age of eldest child} = a)\cdot \I(\text{K$\to$M})_i + X_i\gamma + \varepsilon_i,\]
note that these controls include dummies for the age of the eldest child.

See Figure \ref{pics-diff} for the results. The left panel essentially replicates the differences on the previous graph. The right panel is of the main focus: if the set of controls captures the issues of construction of the groups, then the differences associated with age follow the conjecture above: they are small for younger children and start to hike at ages 5--10 (I believe that first few points are mostly caused by composition). In general, these graphs are kind of consistent with the idea that divorces are concentrated around the moments where kids get older and childcare costs structure changes as they arguably require less time, though they do not provide a formal test.
\begin{figure}[h!]
\includegraphics[width=0.45\linewidth]{diff_raw_by_age.eps}
\includegraphics[width=0.45\linewidth]{diff_cont_by_age.eps}
\caption{Differences in divorce probabilities by child's age\label{pics-diff}}
\end{figure}

\subsubsection{Sample of Recently Divorced}
In this exercise I utilize an alternative variable available in ACS --- indicator whether a particular person got a divorce in the year of survey. It can be interpreted as \textit{marginal} divorce probability as opposed to the previous value being \emph{cumulative}. Therefore, for instance, it does not have an upward trend in number of years married, see Figure \ref{pic-now-aftmar}. 
\begin{figure}[h]
\centering
\includegraphics[width=0.45\linewidth]{divnow_by_aftmar.eps}
\caption{Probabilities of getting divorced in the year of survey\label{pic-now-aftmar}}
\end{figure}

Table \ref{diff-contr-marginal} replicates the results for marginal divorce probabilities. The key conclusions and proportions remain very similar, that implicitly suggests validity of the approach used above.

\begin{table}
\begin{center}
\begin{tabular}{|l|c||c|c|c|c|c|c|}\cline{2-8}
\multicolumn{1}{c|}{} & \multicolumn{7}{c|}{ACS: sample of married once}\\\cline{2-8}
\multicolumn{1}{c|}{} & \small Share of &\multicolumn{2}{c|}{\small Raw difference} & \multicolumn{2}{c|}{\small with controls} & \multicolumn{2}{c|}{\small controls + income} \\ \cline{3-8}
\multicolumn{1}{c|}{} & \small divorced & $\Delta$ & (s.e.) & $\Delta$ & (s.e.) & $\Delta$ & (s.e.) \\  \hline
\textit{All females, 20--50} & 1.94\% &  1.37\% &  (0.05\%) &  0.82\% &  (0.08\%) &  0.79\%  &  (0.11\%) \\
\textit{All females, 35--40} &   1.59\% &  1.04\% &  (0.08\%) &   0.77\% &  (0.11\%)  &  0.74\%   &  (0.15\%) \\\hline\hline
\multicolumn{8}{|p{0.9\linewidth}|}{\footnotesize \textbf{Education partition.}}\\\hline
\textit{High school only, 20--50} & 2.31\% &  0.86\% &  (0.08\%) &   0.82\% &  (0.12\%)  &  0.82\%  &   (0.20\%) \\
\textit{College graduates, 20--50} & 1.19\% &  1.24\% & (0.10\%) &   0.59\% &   (0.13\%)  &  0.50\% &    (0.16\%) \\\hline\hline
\multicolumn{8}{|p{0.9\linewidth}|}{\footnotesize \textbf{Income partition:} working women only.}\\\hline
\textit{Low 30\% income, 20--40} &  3.02\% &  1.31\% &  (0.14\%) &   0.90\% &  (0.21\%) &   0.91\%  &   (0.21\%) \\
\textit{Mid 40\% income, 20--40} &  2.54\% &  1.34\% &  (0.12\%) &   0.54\%  & (0.18\%) &   0.54\% &    (0.18\%) \\
\textit{Top 30\% income, 20--40} & 1.69\% &  1.69\% &   (0.15\%) &   0.90\%  & (0.20\%) &   0.90\% &    (0.20\%) \\
\hline
\end{tabular}
\caption{Differences in marginal probablities, controlled for composition, ACS\label{diff-contr-marginal}}
\end{center}
\end{table}

\subsection{Intra-Household Allocations}

[I look at earnings, hours and LFP differences] [Grade retention depends on child's age, so need to compare given child's age]

[THIS IS AN OLD VERSION OF TABLE]

\begin{center}
\begin{tabular}{|l||c|c||c|c|}\hline
& \multicolumn{2}{|c||}{\footnotesize \textit{All couples}}  & \multicolumn{2}{|c|}{\footnotesize \textit{Top 40\% of income} } \\\hline
&                       \footnotesize  M$\to$K & \footnotesize  K$\to$M &  \footnotesize  M$\to$K & \footnotesize  K$\to$M \\\hline
\footnotesize \textit{Female in Labor Force}        & $70.3\%$  & $71.4\%$  & $79.3\%$ & $85.3\%$  \\\hline
\footnotesize \textit{Earninigs Ratio}    &  $95.2\%$ & $93.9\%$  &  $95.8\%$ &  $96.4\%$ \\\hline
\footnotesize \textit{Kids' Grade Retention}  & $4.6\%$    & $7.6\%$ &   $3.3\%$   & $5.9\%$ \\\hline
\end{tabular}
\end{center}

\section{Model}

% This is part of draft.tex containig the model
% \section{Model}
The model is a mixture between conventional heterogeneous agents savings model (B--A--H framework) and household formation and dynamic bargaining, that is extensively developed by [...]. 

Agents live for $T$ periods. The agents are males, females and married couples. Couples can have a child together, and if a couple with a child divorces up, the child stays with mother, and this is how singe females can have kids. Single females with kids randomly re-enter marriage market by becoming single females without kids.\footnote{The reason for this assumption is to avoid modeling mixture between step and own children, and the fact that single mothers are not a main focus of the paper.}


Single people without kids randomly meet partners with characteristics similar to them and decide whether to create a couple. Couples agree on some marriage terms, represented by each spouse's weight in couple's objective function. The rationale for marriage includes risk-sharing, returns to scale in consumption, possibility to have a child in the future and additional utility surplus of being married (love shock). Each period after random shocks are realized, couples may decide to renegotiate their marriage terms if one of the spouses is better off being single. If they cannot find marriage terms that are satisfactory for both, divorce happens.

Couples can choose to have a child and can also experience unplanned pregnancies. Unplanned pregnancies cannot be aborted. In addition to couples, single people who met each other may experience an unplanned pregnancy just before they decide whether to create a couple. In this case, upon disagreement female becomes a single mother. This mechanism is a stylized representation of kids appearing in unmarried cohabitating couples: since I do not model cohabitation, those kids appear just before people actually decide to marry each other.\textit{ I discuss this issue additionally in Section ....}

\subsection{Remarks on Notation}
I write $V^{i,j}$ to represent individual value functions in the model. Here $i$ represents gender: $f$ for female or $m$ for male. $j$ represents demographic status: $s$ for singles without kids, $sk$ for singles with kids (females only), $c$ for couples without kids and $ck$ for couples with kids. Couple's joint value functions have only one superscript $j \in \{c,ck\}$. Relation between joint value function $V^{c}$ and each spouse's value functions $V^{m,c}$ and $V^{f,c}$ is described in details in couple's section.

\subsection{Singles Without Kids}
Singles without kids are characterized by age $t$, labor productivity $z$ and assets (savings) $a$. I use $\omega = \{z,a\}$ to individual characteristics other than age.

Each period single individuals meet a partner of the same age and characteristics $\omega^p$ with probability $p^{\text{meet}}_t$. If meeting happens, with probability $p^{\text{preg}}_t$ unplanned pregnancy happens. After realizing new period's shocks and pregnancy status couple tries to negotiate marriage terms. If they found marriage terms that are satisfactory for both ($m^{np} = 1$ in case of no pregnancy or $m^p = 1$ in case of pregnancy), they form couple with characteristics $\Omega^c$ . If satisfactory marriage terms could not be found, the individual stays single, or becomes single mother if she is female and unplanned pregnancy happened.

The value function of male agent therefore is
\begin{align}V^{m,s}_t(\omega) = \max\limits_{c} & \bigg\{ u(c) + \beta \E_t \Big[ (1 - p^{\text{meet}}_t)\cdot V^{m,s}_{t+1}(\omega') + \\ \nonumber
& \hspace{7em} p^{\text{meet}}_t (1-p^{\text{preg}}_t) \big\{ m^{np} \cdot V^{m,c}_{t+1}(\Omega^c) + (1-m^{np})V^{m,s}_t(\omega')\big\} + \\  \nonumber
& \hspace{10em} p^{\text{meet}}_t p^{\text{preg}}_t \big\{ m^{p} \cdot V^{m,ck}_{t+1}(\Omega^c) + (1-m^{p})V^{m,s}_{t+1}(\omega')\big\}  \Big]  \bigg\},\\  \nonumber
 \end{align}\vspace{-3em}
 \begin{align*}
 \text{s.t. \ }  &  a' = R\cdot a  + W^m_t(z) - c  & \text{ (evolution of assets)}\\
 &  z' = z + \varepsilon^{z,m}_t, \ \ \varepsilon^{z,m}_t \sim \mathcal{N}(0;\sigma_{z,m}^2) &  \text{ (evolution of productivity)}\\
  & \log W^m_t = z_t + \text{Trend}^m_t, \ \ \  \text{Trend}^m_t = a^m_0 + a^m_1\cdot t  +  a^m_2 \cdot t^2 &  \text{ (labor income and trend)}\\
  & \Omega^c = \mathcal{M}^m(a',z') &  \text{ (marriage prospectives)}
\end{align*}
where random function $\mathcal{M}^m$, representing characteristics of potential couple is defined in Section [...].

The value function of female agent is
\begin{align}V^{f,s}_t(\omega) = \max\limits_{c} & \bigg\{ u(c) + \beta \E_t \Big[ (1 - p^{\text{meet}}_t)\cdot V^{f,s}_{t+1}(\omega') + \label{single-fem} \\  \nonumber
& \hspace{7em} p^{\text{meet}}_t (1-p^{\text{preg}}_t) \big\{ m^{np} \cdot V^{f,c}_{t+1}(\Omega^c) + (1-m^{np})V^{f,s}_t(\omega')\big\} + \\  \nonumber
& \hspace{10em} p^{\text{meet}}_t p^{\text{preg}}_t \big\{ m^{p} \cdot V^{f,ck}_{t+1}(\Omega^c) + (1-m^{p})V^{f,sk}_{t+1}(\omega')\big\}  \Big]  \bigg\},
\end{align}\vspace{-3em}
\begin{align*}
 \text{s.t. \ }  &  a' = R\cdot a  + W^f_t(z) - c  & \text{ (evolution of assets)}\\
 &  z' = z + \varepsilon^{z,f}_t, \ \ \varepsilon^{z,f}_t \sim \mathcal{N}(0;\sigma_{z,f}^2) &  \text{ (evolution of productivity)}\\
  & \log W^f_t = z_t + \text{Trend}^f_t, \ \ \  \text{Trend}^f_t = a^f_0 + a^f_1\cdot t  +  a^f_2 \cdot t^2 &  \text{ (labor income and trend)}\\
  & \Omega^c = \mathcal{M}^f(a',z') &  \text{ (marriage prospectives)}
\end{align*}

% 
The crucial difference here is the term $V^{f,sk}$: it reflects the fact that upon disagreement woman becomes a single female with a child. 

\subsection{Couples Without Kids}
Couple is characterized by age $t$ (assumed the same for both spouses), marriage terms $\theta$, productivities of female and male $z^f$, $z^m$ and additive marriage surplus $\psi$. Couple maximizes weighted expected lifetime utility, and $\theta$ represents share of female in this objective function, therefore $\Omega = \{\theta,\psi,z^f,z^m\}$ represent characteristics of couple.

The key element shaping couple's decisions is participation constraints: each period the expected lifetime utility of staying in couple should be not less that expected lifetime utility of getting a divorce and becoming a single agent. Depending on realization of the shocks, if it is possible to satisfy both participation constraints the next period (so no divorce happens, $d = 0$) state is $\Omega'$. If participation constraints cannot be satisfied, the couple gets a divorce $d = 1$ and its characteristics are dissolved onto $\omega^{df}$ and $\omega^{dm}$.  In more details this is described in Section [...].

Fertility dimension is an important element. Each period couples get fertility shock $p^{\text{preg}}_t$. If it happens, they become couple with a child if they stay together (renegotiation happens after they know realization of the shock), or become single male and single female with a child if they do not. If the shock does not happen, they still may make a decision to get a baby \emph{after} they renegotiate the marriage terms if they stay together. This timing is subtle, but is required for everything to be internally consistent, see Appendix ... for more details on how timing interacts with renegotiation. 

After realization of all the shocks and making fertility and divorce decisions in the current period the problem of the couple that stays childless is
\begin{align}& \hspace{5em}  V^{c}_t(\Omega) = \max\limits_{c^f,c^m}  \bigg\{ \theta\cdot u(c^f) + (1-\theta)\cdot u(c^m) + \psi +  \label{vf-c} \\   \nonumber
 &  \beta \E_t \Big[   (1 - p^{\text{preg}}_t)\cdot \left\{ (1-d^{\text{np}})\cdot \max\left\{ V^{ck}_{t+1}(\Omega'),V^{c}_{t+1}(\Omega')\right\} + d^{\text{np}}\cdot [ \theta V_{t+1}^{f,s}(\omega^{df}) + (1-\theta)V_{t+1}^{m,s}(\omega^{dm})]\right\}  +  \\  \nonumber
& \hspace{5em} p^{\text{preg}}_t\cdot \left\{ (1-d^{\text{p}})\cdot V^{ck}_{t+1}(\Omega') + d^{\text{p}}\cdot [ \theta V_{t+1}^{f,sk}(\omega^{df}) + (1-\theta)V_{t+1}^{m,s}(\omega^{dm})]\right\} \Big] \bigg\},
\end{align}\vspace{-2em}
\begin{align*}
\text{s.t. \ }& a' + c = R\cdot a  + W^m_t(z^m) + W^f_t(z^f) & \text{(evolution of joint assets)},\\
				 & c = [(c^f)^{1+\rho_c} + (c^m)^{1+\rho_c}]^{\frac1{1+\rho_c}} & \text{(increasing returns in consumption)},\\
				 &  z^{f\prime} = z^f + \varepsilon^{z,f}_t, \ \ \varepsilon^{z,f}_t \sim \mathcal{N}(0;\sigma_{z,f}^2) &  \text{ (evolution of female productivity)}\\
				 &  z^{m\prime} = z^m + \varepsilon^{z,m}_t, \ \ \varepsilon^{z,m}_t \sim \mathcal{N}(0;\sigma_{z,m}^2) &  \text{ (evolution of male productivity)}\\
                    & \psi' = \psi + \varepsilon^{\psi}_t, \ \ \varepsilon^{\psi}_t \sim \mathcal{N}(0;\sigma_{\psi}^2)  & \text{(evolution of marriage surplus),} \\
                    & (\Omega',\omega^{df},\omega^{dm}) = \mathcal{R}(\theta,a',z^{m\prime},z^{f\prime},\psi') & \text{(renegotiation correspondence)},
\end{align*}

\subsubsection{Defining Individual Values}
The couple's collective value function $V^c$ is the object relevant for making decisions of consumption and fertility. Making marriage and divorce decisions involves individual value functions of spouses living in the couple. I denote them as $V^{f,c}$ and $V^{m,c}$. These value functions are obtained by plugging optimal decisions of couple into intertemporal utilities of each agent and accounting for possible transitions. Note that they cannot be derived from a maximization problem, and common properties of value functions such as envelope theorems do not hold for them.

Namely, given couple's decisions $c^f(\Omega)$, female's value of being in couple is defined recursively as
\begin{align}
& \hspace{5em}  V^{f,c}_t(\Omega) =    u(c^f) + \psi +  \\   \nonumber
 &  \beta \E_t \Big[   (1 - p^{\text{preg}}_t)\cdot \left\{ (1-d^{\text{np}})\cdot \max{}^C \left\{ V^{f,ck}_{t+1}(\Omega'),V^{f,c}_{t+1}(\Omega')\right\} + d^{\text{np}}\cdot [ V_{t+1}^{f,s}(\omega^{df})]\right\}  +  \\  \nonumber
& \hspace{5em} p^{\text{preg}}_t\cdot \left\{ (1-d^{\text{p}})\cdot V^{f,ck}_{t+1}(\Omega') + d^{\text{p}}\cdot [ V_{t+1}^{f,sk}(\omega^{df}) ]\right\} \Big] 
\end{align}
where operator $\max{}^C\{A,B\} \equiv A\cdot \I[V^{ck}_{t+1}(\Omega')\geq V^{c}_{t+1}(\Omega')] + B\cdot \I[V^{ck}_{t+1}(\Omega')< V^{c}_{t+1}(\Omega')]$. It reflects the fact that couple makes joint decision that is not necessary individually optimal.

Another important property of individual values is that
\begin{equation} V^{c}_t(\Omega) \neq \theta\cdot V^{f,c}_t(\Omega) + (1-\theta)\cdot V^{m,c}_t(\Omega),\label{tht_noneq}\end{equation}
the main reason for this is that $\theta$ changes in the future as a result of random shocks, and this change is not likely to be symmetric. This, in particular, causes couple's expected future value function to be discontinuous in $(a,z)$. See Appendix [...] for some additional discussion.

\subsection{Couples With Kids}
State space for couples is the same as for singles, except for additional variable $\xi$ representing child's age, therefore $\Omega = \{\theta,\psi,z^f,z^m,\xi\}$. I consider just two states for $\xi \in \{y,s\}$ representing young and school-age child. At birth the child is young, then each period with probability $p^{s}$ the child becomes of school-age. The difference between young and school-age children are mother's time requirement, namely young children require more time and elder --- more money.

Couples with kids have additional choice variable $q$, that can be interpreted as flow of child quality or child consumption. Child quality is produced using mother's labor $l_f$ and monetary expenditures $x$ according to constant returns to scale production function. Labor input to child quality causes female to lose part of her labor income. The production function depends on age, namely I parametrize it as $q = [\mu_\xi\cdot l_f^{\theta_q} + (1-\mu_\xi)\cdot x^{\theta_q}]^{\frac1{\theta_q}}$. This follows, for instance, [Sommer ...], with the addition of changing labor share $\mu$.

\begin{align}& \hspace{5em}  V^{ck}_t(\Omega) = \max\limits_{c^f,c^m,q,l_f,x}  \bigg\{ \theta\cdot u(c^f,q) + (1-\theta)\cdot u(c^m,q) + \psi + \label{vf_ck} \\  \nonumber
 & \hspace{10em} \beta \E_t \Big[   \left\{ (1-d)\cdot   V^{ck}_{t+1}(\Omega') + d\cdot [ \theta V_{t+1}^{f,sk}(\omega^{df}) + (1-\theta)V_{t+1}^{m,s}(\omega^{dm})]\right\} \Big] \bigg\},
\end{align}\vspace{-2em}
\begin{align*}
\text{s.t. \ } & a' + c + x = R\cdot a  + W^m_t(z^m) + (1-l_f)\cdot W^f_t(z^f) & \text{(evolution of joint assets)},\\
                    & c = [(c^f)^{1+\rho_c} + (c^m)^{1+\rho_c}]^{\frac1{1+\rho_c}} & \text{(increasing returns in consumption)},\\
                    & q = [\mu_\xi\cdot l_f^{\theta_q} + (1-\mu_\xi)\cdot x^{\theta_q}]^{\frac1{\theta_q}} & \text{(production function of child quality)},\\
                    &  z^{f\prime} = z^f + \varepsilon^{z,f}_t, \ \ \varepsilon^{z,f}_t \sim \mathcal{N}(0;\sigma_{z,f}^2) &  \text{ (evolution of female productivity)}\\
				 &  z^{m\prime} = z^m + \varepsilon^{z,m}_t, \ \ \varepsilon^{z,m}_t \sim \mathcal{N}(0;\sigma_{z,m}^2) &  \text{ (evolution of male productivity)}\\
                    & \psi' = \psi + \varepsilon^{\psi}_t, \ \ \varepsilon^{\psi}_t \sim \mathcal{N}(0;\sigma_{\psi}^2)  & \text{(evolution of marriage surplus),} \\
                   &  \P(\xi' = s | \xi = y) = p^s, \ \ \P(\xi' = s | \xi = s) = 1 & \text{(evolution of child's age)},\\
                    & (\Omega',\omega^{df},\omega^{dm}) = \mathcal{R}(\theta,a',z^{m\prime},z^{f\prime},\psi',\xi) & \text{(renegotiation correspondence)},
\end{align*}

\subsection{Single Mothers}
State space for single mothers contains individual productivity and child's age. Single mothers cannot marry, but have chance $p^{\text{out}}$ of recovering their marriage prospectives. In the model this is captured by transition of them to single females. This form is somewhat restrictive, although any other option requires modeling marriage of single mothers separately. This is a complex issue, for instance, males would seek females who already have kids if we do not make difference between step and own children. Since for the purposes of the model option of being single mother is used mainly as a factor affecting bargaining, I introduce additional (constant) utility factor $S$ to flexibly control utility of exiting the state. This can be interpreted as equivalent of utility of kids from past marriages when entering the marriage market again. This form seems flexible enough to capture variation in an option of being a single mother without putting extra computational and modeling burden.

The problem of a female who stays single mother is
\begin{align}V^{f,sk}_t(\omega) = \max\limits_{c} & \bigg\{ u(c,q) + \beta \E_t \Big[ p^{\text{out}} \cdot \left\{ V^{f,s}_{t+1}(\omega') + S\right\} +(1-p^{\text{out}})V^{f,sk}_{t+1}(\omega') \Big]  \bigg\},
\end{align}\vspace{-1em}
\begin{align*}
 \text{s.t. \ }  &  a' = R\cdot a  + (1-l_f)W^f_t(z) - c  & \text{ (evolution of assets)}\\
 & q = [\mu_\xi\cdot l_f^{\theta_q} + (1-\mu_\xi)\cdot x^{\theta_q}]^{\frac1{\theta_q}} & \text{(production function of child quality)},\\
 &  z' = z + \varepsilon^{z,f}_t, \ \ \varepsilon^{z,f}_t \sim \mathcal{N}(0;\sigma_{z,f}^2) &  \text{ (evolution of productivity)}\\
 &  \P(\xi' = s | \xi = y) = p^s, \ \ \P(\xi' = s | \xi = s) = 1 & \text{(evolution of child's age)},\\
\end{align*}

\subsection{Marriage Market}
Single males and females meet potential partners of identical age and characteristics that depend on their assets $a$ (chosen in the previous period) and productivity $z$ (revealed after realization of shock):
\begin{equation}\log a^p = \log a + \varepsilon^{a,p}, \ \ \varepsilon^{a,p} \sim \mathcal{N}(0,\sigma_{a,p}^2),\end{equation}
\begin{equation}	z^p = z + \varepsilon^{z,p}, \ \ \varepsilon^{z,p} \sim \mathcal{N}(0,\sigma_{z,p}^2),\end{equation}
for example, $\sigma_{a,p} = 0.1$ can be read as standard deviation of partner's assets being 10\% around of what agent has. 
Additionally, initial marriage surplus (love shock) is drawn from
\begin{equation} \psi \sim \mathcal{N}(0,\sigma_{\psi,0}^2).\end{equation}

If people agree to be a couple they pool their assets $a^c = a + a^p$. If no unplanned pregnancy happens, people agree to marry if set of mutually satisfactory marriage terms
\begin{equation}\small \Theta^{np}_t = \left\{ \theta : V_t^{f,c}(\theta,\psi,a^c,z^f,z^m) \geq V_t^{f,s}(a^f,z^f), \ \ V_t^{m,c}(\theta,\psi_0,a^c,z^f,z^m) \geq V_t^{m,s}(a^m,z^m)\right\}\end{equation}
is non-empty. If potential couple gets the pregnancy shock, the set changes to
\begin{equation} \small \Theta^{p}_t = \left\{ \theta : V_t^{f,ck}(\theta,\psi,a^c,z^f,z^m,y) \geq V_t^{f,sk}(a^f,z^f,y), \ \ V_t^{m,ck}(\theta,\psi_0,a^c,z^f,z^m,y) \geq V_t^{m,s}(a^m,z^m)\right\} \end{equation}
(at birth child's age is $\xi = y$).

If the respective set is non-empty, initial bargaining power of couple is determined by symmetric Nash Bargaining, for instance:
\small
\begin{align}&\theta^*(a,z,\psi,\epsilon^{a,p},\epsilon^{z,p}) = \label{nbs} \\ \nonumber &= \argmax\limits_{\theta \in \Theta^{np}} \left[V_t^{f,c}(\theta,\psi,a^c,z^f,z^m) - V_t^{f,s}(a^f,z^f)\right] \times \left[V_t^{\vphantom{f}m,c}(\theta,\psi,a^c,z^f,z^m) - V_t^{m,s}(a^m,z^m)\right],\end{align}
and analogously for the case when pregnancy shock happened.

From the prospective of a single agent, this set is determined by realization of $\psi$ and partner's characteristics $\epsilon^{a,p}$ and $\epsilon^{z,p}$. Given them, marriage function $m$ is binary:
\[m^{np}_t(a,z,\psi,\epsilon^{a,p},\epsilon^{z,p}) = \I(\Theta^{np}_t \neq \varnothing),\]
where variables affecting set $\Theta$ are unambiguously recovered from arguments of $(a,z,\psi,\epsilon^{a,p},\epsilon^{z,p})$ as described above.

%Because of good properties of Nash Bargaining, expected future value function is continuous with respect to of $a$ and $z$. See Appendix [...] for additional discussion of this.

Let $\omega = (a,z)$ and $\epsilon = (\psi,\epsilon^{a,p},\epsilon^{z,p})$. This allows to define a random function
\[\Omega^c = \mathcal{M}_t(\omega) = M_t(\omega,\epsilon) = (\theta^*,\psi,a^c,z^m,z^f),\]
so the future characteristics of couple are function of current characteristics of individual and three-dimensional random shock.
\subsection{Renegotiation And Divorce}

Upon divorce, $\kappa\cdot a$ of assets disappears and the rest is split evenly. This generalizes [Voena ...], where I put additional parameter $\kappa$ captures can rationalize extra costs of divorce for more wealthy couples. [Voena ...] argues that this approximates well actual property divisions. So, the splitting rule is
\[a^{df} = 0.5\cdot (1-\kappa)\cdot a,  \ \ a^{dm} = 0.5\cdot (1-\kappa)\cdot a.\]
This splitting rule defines mapping of couple's state $\Omega$ to individual states $\{\omega^{df},\omega^{dm}\}$ in case of divorce. 

The option to leave with a share of couple's assets creates pressure for participation constraints. As argues by [...], the household prefers to keep $\theta$ constant. However, after shocks to $z$ and $\psi$ happen, one (or both) participation constraints might be violated. To address this formally, consider couple that already has kids. I define set
\begin{equation}\Theta_t = \left\{ \tilde\theta : V^{f,ck}_t(a,\tilde\theta,\psi,z^f,z^m,\xi) \geq V_t^{f,sk}(a^{df},z^f,\xi) , \ \ V_t^{m,ck}(a,\tilde\theta,\psi,z^f,z^m,\xi)\geq V_t^{m,s}(a^{dm},z^m) \right\} \label{reneg-with-kids}
,\end{equation}
and if couple starts with current value $\theta_{0}$ there are three possible options:
\begin{enumerate}
\item \textit{Status quo.} If $\theta_0 \in \Theta_t$, current marriage terms are satisfactory for both spouses so $\theta_t = \theta_{0}$.
\item \textit{Renegotiation.} If $\theta_0 \not\in \Theta_t$, but $\Theta_t \neq \varnothing$, then spouses can continue to be married if they change their marriage terms. In this case they pick the closest the new value from set $\Theta_t$ such that it is the closest to the old marriage terms: $\theta_t = \min\limits_{\tilde\theta\in\Theta_t} |\tilde\theta - \theta_0|$. 
\item \textit{Divorce.}  If $\Theta_t = \varnothing$, so if there do not exist any marriage terms satisfactory for both spouses, then divorce happens according to the rule described above.
\end{enumerate}
Picking $\theta$ that is the closest to the old one is not a random choice: it is rationalized by models with limited commitment models, where change in $\theta$ represents the value of Lagrange multiplier on binding participation constraint. Intuitively, since couple does not like $\theta$ to be changing and everything is continuous in $\theta$, the couple prefers small changes to larger ones, and this drives the result.

Renegotiation that involves making discrete decisions has few more issues as backward induction has to be used: fertility decisions have to be made after new $\theta$ is set, and these decisions might depend on $\theta$, therefore inequalities defining set $\Theta_t$ need to be modified. This is an artifact of assuming that fertility decisions are made collectively and it can be the case that having a baby benefits couple but hurts wife alone (mainly by worsening her outside option), and such scenario of renegotiation insures that participation constraints hold after the fertility decisions are made. See Appendix \ref{ren-disc} where I elaborate on this.


\subsection{Additional Details}
The model I solve an estimate contains few additional things that are omitted for more clear exposition above.

First, females may exit the state of being single mother immediately. This transition is assumed to happen after divorce or failure to agree to marry, therefore in the value functions and in (re)negotiation I use $V^{f,s?} = p^{\text{out}} [V^{f,s} + S] + (1-p^{\text{out}})V^{f,sk}$ instead of just $V^{f,sk}$. This allows for quicker re-marriage and easier implementation of counterfactual in which females do not have risk of becoming single mother at all. I use notation of $V^{f,s?}$ in Appendix \ref{add-trans} where I show exact value functions.

Second, I assume that having a baby is possible only before certain age $\bar{T}$. After that, value functions and negotiation rules are modified in a straightforward manner.

Finally, to simplify computations I also do not allow marriage status and marriage terms to change after certain $\bar{T}_2$. Therefore I drop participation constraints in the couples, the value functions look the same but do not allow for $\theta$ to change.

\subsection{Functional Forms}
I use standard CES utility functions for individuals and couples with no kids: $u(c) = \frac{c^{1-\sigma}}{1-\sigma}$.  When kids appear, people extract additional utility of child quality, I use the following non-separable utility function:
\[u(c,q) = \frac{\left(e^{\phi_0}\cdot c^{1-\phi_1} \cdot (\phi_2 + q)^{\phi_1}\right)^{1-\sigma}}{1-\sigma},\]
parameter $\phi_0$ represents fixed gain in marginal utility because of having a baby, $\phi_1$ drives relative importance of child quality in utility function and $\phi_2$ is a ``luxury good'' parameter that can rationalize large child expenditures of wealthy couples that are discussed in the estimation section.

\section{Estimation Strategy}
I use techniques of indirect inference, extensively described, for instance, in [...] for estimation of the parameters. Shortly, I select parameters such that features of the real data are similar to the features of the data simulated from model. It is not entirely correct to call the procedure simulated method of moments, as some things that I match are not moments in econometric sense, however it has many similarities with simulated method of moments so I occasionally extend notion of moments to reflect any statistics derived from the data.

\subsection{Overview}
Full list of parameters involves many things:
\begin{itemize}
\item Fundamental parameters of income fluctuation model $(\beta,\sigma,R)$
\item Intra-household returns to scale $\rho_c$
\item Preferences for child's quality: $(\phi_0,\phi_1,\phi_2)$
\item Parameters for child's quality production function: CES power $\theta_k$ and labor shares by age $(\mu_y,\mu_s)$.
\item Transition probabilities:
\begin{itemize}
\item Probability of pregnancy shock by age: $p^{\text{preg}}_t = \lambda^{\text{preg}}_0 - \lambda^{\text{preg}}_1\cdot t$ (2 parameters)
\item Probability of meeting a partner: $p^{\text{meet}}_t:$ piecewise linear function between $(p^{\text{meet}}_{21},p^{\text{meet}}_{27},p^{\text{meet}}_{34},p^{\text{meet}}_{40})$ (4 parameters)
\item Probability of re-entering marriage market and corresponding utility ``compensation'' $(p^{\text{out}},S)$. (2 parameters)
\item Probability that kids grow up (transition from $y$ to $s$): $p_s$ (1 parameter)
\end{itemize}
\item Age profiles of labor income by gender: $(a^f_0,a^f_1,a^f_2)$, $(a^m_0,a^m_1,a^m_2)$ (6 parameters)
\item Variance of shocks (6 parameters):
\begin{itemize}
\item Individual income shocks $\sigma^{z,f}$, $\sigma^{z,m}$.
\item Couple's surplus shocks $\sigma^{\psi}$
\item Couple's initial surplus $\sigma^{\psi_0}$
\item Assets and productivity of potential partner: $(\sigma^{z,p},\sigma^{a,p})$
\end{itemize}
\end{itemize}

Total number of parameters here is 31. Few additional choices involve time horizons, bounds for random walk variables and bounds for assets and $\theta$.



\subsection{Externally Set Parameters}
Length of the period is set to be 1 year, the model assumes people start their life with no assets at the age of $21$ and die deterministically at the age of $81$ (so $T = 30$). However, it assumes that fertility status can change only till age of 40 ($T_f = 20$) and marriage status till age of 45 ($T_m = 25$). It also assumes that after the age of 60 people retire and get $b = 0.4$ [as in Sommers, 2014] of their income at age of 60 (so no transitions happen). I only simulate the data for age 21--40, therefore these choices are not likely to play a crucial role. 

Following standard approach in literature, I fix $\sigma = 1.5$, $R = 1.03$ and $\beta = 1/R$. %Also, following estimates of Voena (2014) I set $\rho_c = 0.23$. 

I take age-income profiles  $(a^f_0,a^f_1,a^f_2)$, $(a^m_0,a^m_1,a^m_2)$ to match coefficients in per-hour labor earnings in ACS (see Appendix [...] for details). I borrow variances of income shocks $\sigma^{z,f}$, $\sigma^{z,m}$ from Low et al 2018 as they cannot be estimated in cross-sectional data [This could be fixed in later versions as their variances are for subsample of non-college graduates]. 

I also externally estimate $\sigma^{z,p}$ as variance of (unpredicted part of) log-difference in spouses earnings, see Appendix \ref{partearn} for this. Based on this calculations I set $\sigma^{z,p} = 0.8$. As I both do not have good assets data and do not aim to replicate assets distribution, I set $\sigma^{a,p} = \sigma^{z,p}$.

I also set $p_s = \frac14$ to roughly reflect that children go to school on average at 6 and to pre-school few years earlier and this should correspond to a change in childcare costs structure.

In the part with assets, I assume savings can be between 0 and 20 times average labor income in the beginning of life. This bar in pretty low, but it is enough to create reasonable assets variation at the moment most of marriage decisions are made: say, median assets-to-income ratio at 30 is around 1 [see ...]. 

\subsection{Estimated Parameters}
Out of 31, 15 parameters are set externally, so I am left with $16$ parameters to estimate: child's parameters $(\phi_0,\phi_1,\phi_2)$, child quality production parameters $(\theta_k,\mu_y,\mu_a)$, variances of initial marriage surplus and its innovations $(\sigma^{\psi_0},\sigma^{\psi})$, meeting probability parameters $(p^{\text{meet}}_{21},p^{\text{meet}}_{27},p^{\text{meet}}_{34},p^{\text{meet}}_{40})$, pregnancy probability parameters $(\lambda^{\text{preg}}_0,\lambda^{\text{preg}}_1)$, transition out of being single mother parameters $p^{\text{out}}$, $S$. Unfortunately, all of them interact and as the model is highly stylized but involves many choices, they cannot be estimated externally. 

Conceptually, they can be split onto following groups:
\begin{enumerate}
\item Unplanned pregnancy probabilities are pinned down by share of $K\to M$ females at ages 25 and 35, as well as share of planned births in total population
\item Meeting probabilities and $\sigma^{\psi_0}$ are pinned down by shares of never married females people at 25, 35, 40 and share of females with one marriage at 40 (relevant to capture chances of re-marrying).
\item  Preferences for kids are captured by share of childless females by age of 25, 30, 35 (identifies general scale $\phi_0$), share of childless females for low and high income at 30 $\phi_1$. Luxury good parameter $\phi_2$ is pinned down by ratio of amounts of money spent on kids between 75 and 25 income quantiles.
\item Childcare production function is recovered by ratio of time spent with kids by couple's income (pins down elasticity of substitution), average share of money spent on kids (captures ``average'' $\mu$) and ratio of time spent by child's age (captures difference between $\mu_a$ and $\mu_y$)
\item Variance of innovation to marriage surplus is mainly pinned down by share of divorced people by age (25, 35, 40).
\item Transition out of single mothers is captured by share of never married single mothers at 25 (picks $S$) and share of divorced women with kids at 30 (picks $p^{\text{out}}$).
\item Returns to scale in marriage $\rho_c$ are pinned down by comparing divorce probabilities between high and low income females.
\end{enumerate}
On top of it, I use shares of divorced people for $K\to M$ and $M \to K$ at 40 and their ratio by income groups as additional targets that are affected by mixture of parameters described above.

For income considerations I trim females/couples with $l$ larger than its 77\% percentile (corresponding to $77\%$ of ACS females of 21--40 who work at least 5 hours, imputing $l = 0$ for those who does not have kids) in simulated data, and treat their log per-hour income as missing. I then perform similar detrending and divide individuals by group using the same regression procedure. I prefer detrending to simply picking $z$ as choices of $l$ are not independent on age.
  

\section{Results}
Table \ref{targets-main} shows the targets in the model and in the data. [Marriage dimension is somewhat off, this is not crucial as the study mainly contains married couples, though I have few ideas how to fix it in finite time.]
\begin{table}
\begin{center}
\begin{tabular}{|l|c|c|}\hline
Target & Model & Data \\\hline
\multicolumn{3}{|c|}{Marriage Dimension} \\\hline
Never married by 25  & $32.8\%$ & $68.3\%$  \\
Never married by 30 & $12.6\%$ & $25.6\%$  \\
Never married by 40 & $6.3\%$ & $18.8\%$  \\\hline
One marriage by 40 & $66.3\%$ & $64.7\%$ \\\hline
\multicolumn{3}{|c|}{Fertility Dimension} \\\hline
No kids at 25 & $59.0\%$ & $67.7\%$ \\
No kids at 30 (low $30\%$ income) & $46.7\%$  &   $44.0\%$ \\
No kids at 30 (top $30\%$ income) & $52.1\%$  &   $57.8\%$ \\
No kids at 35  & $32.9\%$  &   $28.1\%$ \\
No kids at 40 & $17.6\%$ & $25.6\%$ \\\hline
\multicolumn{3}{|c|}{Marriage/Fertility Interactions} \\\hline
Never married with kids at 30 & $7.0\%$  & $12.3\%$ \\
Divorced with kids at 30 & $5.7\%$ & $3.8\%$ \\\hline
\multicolumn{3}{|c|}{Divorce Dimension (Divorced = Divorced and not remarried)} \\\hline
Divorced by 25 & 6.6\%   &  10.7\% \\
Divorced by 30 (low $30\%$ income)  & 14.5\% &     20.8\% \\
Divorced by 30 (top $30\%$ income)  & 14.0\% &     8.1\% \\
Divorced by 35  & 15.8\% &    15.9\% \\
Divorced by 40  & 15.5\% &    19.1\% \\\hline
\multicolumn{3}{|c|}{K$\to$M --- M$\to$K partition} \\\hline
Divorced by 40 if K$\to$M & 22.1\% &   23.5\% \\
Divorced by 40 if M$\to$K & 13.8\% &      13.0\% \\
Divorced by 40 if K$\to$M and low 30\% & 34.0\% & 28.5\% \\
Divorced by 40 if K$\to$M and top 30\% & 28.9\% & 23.3\% \\
Divorced by 40 if M$\to$K and low 30\% & 34.0\% & 28.5\% \\
Divorced by 40 if M$\to$K and top 30\% & 19.3\% & 11.3\% \\
Divorced by 40 if M$\to$K and low 30\% & 28.7\% & 18.1\% \\\hline
\multicolumn{3}{|c|}{Childcare} \\\hline
Money for childcare / labor income & 25.7\% & 40.0\% \\
Time spent 0--5 / Time spent 6--12 & 1.92 & 2 \\
Time spent top 30\% / Time spent low 30\% & 0.92 & 1.19 \\
Money spent top 30\% / Money spent low 30\% & 2.09 & 2.67 \\\hline

\end{tabular}
\caption{Targets in the data and in the model\label{targets-main}}
\end{center}
\end{table}


\subsection{Counterfactual Simulations}
[TBD]
\begin{itemize}
\item Turn off fertility completely
\item Turn off bargaining
\item No pressure of being single mother
\item Single mother/single father randomly
\end{itemize}

\subsubsection{Altering Fertility Shocks}
I consider two counterfactual scenarios: when unexpected fertility never happens at the moment people meet and when it never happens at all (but people still can make kids by choice). I compare key targets in Table \ref{cf-shocks}. In general, it looks like contribution of fertility shocks to the divorce rates is quite substantial, and major part of it is driven by fertility shocks before the marriage happens. Especially reducing fertility shocks affects divorces in early stages.
\begin{table}
\begin{center}
\begin{tabular}{|l|c|c|c|}\hline
Target & Baseline & No shocks at meeting & No shocks at all \\\hline
\multicolumn{4}{|c|}{Marriage Dimension} \\\hline
Never married by 25  & $32.8\%$ & $30.6\%$ & $31.1\%$  \\
Never married by 30 & $12.6\%$ & $10.8\%$ & $11.2\%$  \\
Never married by 40 & $6.3\%$ & $5.7\%$ & $6.1\%$  \\\hline
\multicolumn{4}{|c|}{Fertility Dimension} \\\hline
No kids at 25 & $59.0\%$ & $66.3\%$ & $68.3\%$ \\
No kids at 30 (low $30\%$ income) & $46.7\%$  &   $53.5\%$ & $54.8\%$ \\
No kids at 30 (top $30\%$ income) & $52.1\%$  &   $61.9\%$ & $62.7\%$ \\
No kids at 35  & $32.9\%$  &   $36.2\%$ & $36.6\%$ \\\hline
\multicolumn{4}{|c|}{Divorce Dimension} \\\hline
Divorced by 25 & 6.6\%   &  2.6\% & 1.4\% \\
Divorced by 30 (low $30\%$ income)  & 14.5\% &     11.0\% & 8.6\% \\
Divorced by 30 (top $30\%$ income)  & 14.0\% &     9.0\% & 7.5\% \\
Divorced by 35  & 15.8\% &    13.6\% & 11.9\% \\
Divorced by 40  & 15.5\% &    14.3\% & 13.0\% \\\hline
\end{tabular}
\caption{Contribution of fertility shocks \label{cf-shocks}}
\end{center}
\end{table}









\section{Robustness Checks}
\section{Conclusions}
\clearpage
\section*{Literature}
\bibliography{bib}


\newpage
\appendix
\section{Additional Empirical Results}
\subsection{Alternative Income Partitions\label{inc-part}}

Defining income groups in cross-sectional comparisons is no trivial, as income is endogenous to household's composition that is the main object of study here. I in this section I build several altrenative income partitions and argue that qualitatively pattern I care about holds for variety of definitions, although exact numbers may differ quite a lot.



\subsubsection{Equivalence Scale}

Following development and household literature, when dividing people by income group I use total household income adjusted by OECD equivalence scale. The scale assigns weight 1 to the household head, 0.5 to spouse and 0.3 to each child.

As I use the household income as the main variable of interest, I have to restrict attention to couples that live by themselves and single females with children. I do this both for ACS and SIPP, and as definition of households wary a little, this may drop out slightly different groups of people.

Formally,for income considerations I take a subsample of females who is marked as household's head or head's spouse, and who do not have any people living in household other than their spouse/unmarried partner and their children. This, for instance, excludes multigenerational household and people living with roommates, as total household's income may not be a relevant thing for some types of household composition. I cannot define income equivalence scale for more complicated living arrangements, although there are different scales that allow to do this.

To mitigate selectivity, I compute income percentiles in all sample of females with household composition mentioned above, regardless of their marriage and fertility status. I use total family income as the key variable. To mitigate misreporting I drop observations with total yearly family income below $\$4000$ for ACS (that is about a third of federal the poverty level) or monthly income below $\$300$ for SIPP.

[TBD]

\section{Additional Robustness Checks}

\section{Additional Notes on Model}
\subsection{Continuity of Value Functions}
This part discusses violations of continuity caused by the dynamic bargaining structure. The general conclusion is that collective value function does is not necessary continuous in couple's characteristics, and this discontinuity propagates onto all value functions in the model.

To simplify the example, let's consider value function of couple with kids. Similarly to equation \ref{tht_noneq}, we can write
\[ V^{ck}_t(\Omega) = \theta\cdot V^{f,ck}_t(\Omega) + (1-\theta)\cdot V^{m,ck}_t(\Omega) + \Delta(\Omega),\]
and use it to substitute the term under the expectation in equation \ref{vf_ck} 
\begin{align*} & (1-d)\cdot   V^{ck}_{t+1}(\Omega') + d\cdot [ \theta V_{t+1}^{f,sk}(\omega^{df}) + (1-\theta)V_{t+1}^{m,s}(\omega^{dm})] =    V^{ck}_{t+1}(\Omega') + \\
 &  \ \ \ \theta \cdot d(\Omega') \cdot [V_{t+1}^{f,sk}(\omega^{df}) - V_{t+1}^{f,ck}(\Omega')] + (1-\theta) \cdot d(\Omega') \cdot [V_{t+1}^{m,sk}(\omega^{dm}) - V_{t+1}^{m,ck}(\Omega')] - d(\Omega')\cdot \Delta(\Omega'),
\end{align*}
if we consider couple that is close to divorce, so there exist two $\Omega'_1$ and $\Omega'_2$  that are close to each other according to some metric, but $d(\Omega'_2) = 1$ and $d(\Omega'_1) = 0$. Note that around the divorce threshold by [...] \textbf{both} spouses should be indifferent between staying on leaving, therefore
\[ V_{t+1}^{f,sk}(\omega^{df}) - V_{t+1}^{f,ck}(\Omega')\approx 0, \ \ V_{t+1}^{m,sk}(\omega^{dm}) - V_{t+1}^{m,ck}(\Omega') \approx 0,\]
therefore if $\Delta(\Omega')$ is non-zero in these points, this whole expression jumps where value of $d$ jumps. [It is not clear whether it is actually zero or not, but in simulations it does look like it is positive.]

\subsection{Renegotiation With Collective Discrete Decisions\label{ren-disc}}
This section explains how discrete decisions interact with couple's renegotiation. The theory requires participation constraints to be satisfied \textit{after} any discrete choice. In the context of fertility decisions, that means that both spouses can predict future decisions and re-negotiate $\theta$ accounting for this prediction (rather than assuming that fertility status does not change).

With slight abuse of notation, let's suppose that couple gets collective value $V^{ck}(\theta,\Omega)$ if it makes a baby and $V^{c}(\theta,\Omega)$ is it does not. Given $\theta$ total value before the decision is made is \[V^{c?}(\theta,\Omega) = \max\{ V^{ck}(\theta,\Omega), V^{c}(\theta,\Omega)\}\]
after that I define
\[V^{f,c?}(\theta,\Omega) = V^{f,ck}(\theta,\Omega)\cdot \I[V^{ck}(\theta,\Omega) \geq V^{c}(\theta,\Omega)] + V^{f,c}(\theta,\Omega) \cdot \I[V^{ck}(\theta,\Omega) < V^{c}(\theta,\Omega)],\]
note that it is not necessary equal to $\max\{ V^{f,ck},V^{f,c} \}$. Similarly we can define $V^{m,c?}(\theta,\Omega)$.

The main thing to notice is that although $V^{c?}$ is continuous in $\theta$ individual values $V^{f,c?}$ and $V^{m,c?}$ are not, unless both spouses always have agreement on whether to make a baby: $V^{ck} \geq V^{c} \Leftrightarrow V^{f,ck} \geq V^{f,c} \Leftrightarrow V^{m,ck} \geq V^{m,c}$. Therefore set
\[\Omega = \left\{\tilde{\theta} : V^{f,c?}(\tilde\theta,\Omega) \geq V^{f,s}(\omega^{df}), \ \  V^{m,c?}(\tilde\theta,\Omega) \geq V^{m,s}(\omega^{dm}) \right\}\]
may be disjoint, and careful approximation of $V$ with respect to $\Omega$ is required. 

The possible solution to it would be to limit fertility decisions to the cases where both spouses agree on having kids, this, however, goes against the logic where couple maximizes collective utility subject to participation constraints. 

\subsection{Accounting for Additional Transitions\label{add-trans}}
In value functions I stated above I assumed that females stays single mother for at least one period if it exits a couple with a kid or if she has a baby at the moment of marriage and disagrees to marry. In the actual model I relax this assumption so the female can stop being single mother immediately. This is needed, for instance, for evaluating counterfactuals in which female can exit couple without pressure of being a single mother. 

I assume that at the moment of (re)negotiation female is unsure about whether she is going to be a single mother. Therefore, I define expected value of leaving the couple to be
\[V_t^{f,c?}(\omega) = p^{\text{out}}[V_t^{f,s}(\omega) + S] + (1-p^{\text{out}})V_t^{f,sk}(\omega),\]
therefore equaiton \ref{single-fem} becomes
\begin{align}V^{f,s}_t(\omega) = \max\limits_{c} & \bigg\{ u(c) + \beta \E_t \Big[ (1 - p^{\text{meet}}_t)\cdot V^{f,s}_{t+1}(\omega') + \label{single-fem-true} \\  \nonumber
& \hspace{7em} p^{\text{meet}}_t (1-p^{\text{preg}}_t) \big\{ m^{np} \cdot V^{f,c}_{t+1}(\Omega^c) + (1-m^{np})V^{f,s}_t(\omega')\big\} + \\  \nonumber
& \hspace{10em} p^{\text{meet}}_t p^{\text{preg}}_t \big\{ m^{p} \cdot V^{f,ck}_{t+1}(\Omega^c) + (1-m^{p})V^{f,s?}_{t+1}(\omega')\big\}  \Big]  \bigg\},
\end{align}
the set is \ref{reneg-with-kids} becomes
\begin{equation}\Theta_t = \left\{ \tilde\theta : V^{f,ck}_t(a,\tilde\theta,\psi,z^f,z^m,\xi) \geq V_t^{f,s?}(a^{df},z^f,\xi) , \ \ V_t^{m,ck}(a,\tilde\theta,\psi,z^f,z^m,\xi)\geq V_t^{m,s}(a^{dm},z^m) \right\} \label{reneg-with-kids}
,\end{equation}
and the analogous term is introduced for initial negotiation of couples in case of pregnancy shock.
 % notes on model
\section{Computational Notes}
This section describes the procedures I use to approximate the solution of the model above. In general, since the model is finite-horizon, value function iteration starting from the last period is used. The value functions and corresponding policy functions are approximated on a grid. Since state space is quite large (it has at least 4 continuous dimensions for couples), I use approximation techniques of [Judd--Maliar--Maliar] to deal with dimensionality. Namely, in the dimension of productivity, marriage terms and marriage surplus the functions are approximated by Smolyak polynomials (that are a variation of multi-dimensional Chebyshev polynomials) on sparse Smolyak grid. However, since the assets dimension is important and non-convex value functions can cause troubles, so I use full grid in the dimension of assets and linearly interpolate between assets points.

An additional issue is integration, required to compute expected future values of the value function accounting for possible (re)negotiation and divorce. This is noted to be the most costly part of value function iteration. I utilize monomial integration rules describe by [JMM] to do it efficiently. The second part of this section describe this. 

\subsection{Value Function Approximation}
I describe techniques I used for value functions, the same principle applies for policy functions: they are just byproduct of value function iteration that I use for simulations.

State space of continuous variables for couples is $\Omega = (a,\theta,\psi,z^f,z^m)$. I partition it as $\Omega = (a,O)$. Smolyak method generates $S$ gridpoints $G = \{O_1,...,O_S\}$, where each $O$ represents some combination of $(\theta,\psi,z^f,z^m)$; and polynomials $P = \{p_1(o),...,p_S(o)\}$. Therefore any function $f(o)$ is approximated by
\[f(o) \approx \hat{f}(o) = \sum_{i=1}^S c_{i}\cdot p_i(o),\]
and coefficients $c$ are found from collocation relation
\[\hat{f}(O_i) = f(O_i) \ \forall i = 1,...,S.\]

Since we have to approximate functions for different levels of $a$, I introduce additional grid $A = \{a_1,...,a_J\}$. For each $a_j$ we can find corresponding coefficients $c$, I denote them $c_i(a_j)$. At points $a_j$ the function can be written as
\[\hat{f}(a_j,o) = \sum\limits_{i=1}^S c_i(a_j) p_i(o),\]
for points between grid $a \not A$ I linearly interpolate $f(a_j,o)$. Generically, to interpolate values for arbitrary $a$ and $o$ the  reasonable approach is to first compute values of $f$ on each point of $A$ grid for current $o$, to get $F = \{f(a_1,o),...,f(a_J,o)\}$ and then interpolate $F$ with respect to $a$. However, I use linear interpolation and since the function is bilinear in $c$ and $p$, this is fully equivalent to linear interpolation of coefficients $c_i(a_j)$. Therefore I define $\hat{c}_i(a)$ to be linearly interpolated value of coefficients, and therefore the function at arbitrary point is approximated by
\[\hat{f}(a,o) = \sum\limits_{i=1}^S \bar{c}_i(a) p_i(o).\]
If we use interpolation other than linear with respect to $a$ this equivalence would be violated so the procedure becomes more complicated as it requires evaluating many values of $f$.

\subsection{Monomial Quadratures}
See [JMM] for a detailed discussion. Let $\epsilon$ be $d$-dimensional normal distribution and we are interested in evaluating $\E(f(\epsilon))$. Standard techinque would involve using Gauss--Hermite quadrature with respect to each dimension of $\epsilon$, this, however, quickly becomes infeasible: if we want to have 5 points with respect to each dimension of 4-dimensional vector we would have $5^d = 625$ points, out of which many will have very low weights. 

Monomial quadratures generate number of integration node that grows much lower than exponent of $d$. I use the one that is proportional to $d^2$. So, the rule generates $Q$ vectors $\{\epsilon_1,...,\epsilon_q\}$ with weights $w_q$, and the integral can just be written as $\E(f(x))\approx \sum_{q} f(\epsilon_q)$.

As it is noted by [...] that there is a separate issue of using quadratures to approximate integrals of discontinuous functions. However, they also show that the model with deterministic transition decisions can be interpreted as a limiting case of model with stochastic (logit) transitions, for which integration by quadratures is valid. I also believe that this can be addressed by re-interpreting shocks to have a discrete distribution with probability mass $w_q$ rather than true normal distribution (so we do not pretend to recover the integrals under true normal distribution). In general, the math background for this does not seem to be well-developed and I use conventional tools at this stage of the project without asserting its mathematical rigor. In the future, I may try to test the approximations I use against just discretizing all random walks with number of points.

\section{Auxiliary Estimates}
\subsection{Partner's Earnings\label{partearn}}
The model assumes that
\[ z^p = z^{\text{own}} + \varepsilon^{z,p}, \ \ \ \varepsilon^{z,p}\sim\mathcal{N}(0,(\sigma^{z,p})^2),\]
so detrended log-wage of potential partner is normally distributed with the mean of own detrended log-wage and standard deviation of $\sigma^{z,p}$.

To estimate $\sigma^{z,p}$ I regress log-difference in spouses earning for couples in ACS on several predictors, that include ages of spouses, ACS year and state. In this section I present several alternative specifications, including an attempt to correct for selection into marriage, and show what variance they imply. 

The main specification is
\[\log \frac{\text{Earnings of wife}}{\text{Earnings of husband}} = \alpha + X'\gamma + \epsilon,\]
where I run regression on subsample of couples in which both spouses work full time and report labor income above 5000 per year.  Controls $X$ include 4th degree polynomials and wife's and husband's age, as well as absolute value of age difference, dummies for states and ACS years. I regress difference in per-hour earnings to the set of controls for couples who married in the year previous to the survey year and who have no children in the household. Relaxing this definition, using total earnings instead of per-hour and even controlling for education lead to slightly different estimates, but all of them are within $[0.66,0.8]$ range.

I pick the upper bound of this interval and set $\sigma^{z,p} = 0.8$: as in the model people are more likely to agree to marry similar partners, I expect similar statistic in the simulations to be slightly lower. [Similar statistic computed on simulated data returns ...]

 % notes on computations
\newpage
\section*{Very Technical Appendix, Not For Final Version}


\section{Expressions for Value Functions}
\subsection{Notation}
This technical note shows exact expressions for approximating value functions. For the couple's state I use partition $\Omega = (a,O)$, where $O = (\theta,\psi,z^f,z^m)$. For the age variable I put $\xi\in\{y,s\}$ counts as a separate letter, so states for couples with kids are noted like $cky$ and $cks$. Let $p^s$ represents probability that the child ``grows up'' so couple switches from $cky$ to $cks$. For individual's state there are only two variables $\omega = (a,z)$, and child's age for single mother is reported as state $sky$ if $\xi = y$ and $sks$ if $\xi = y$. 

Grid consists of all possible combinations of $a$ and $O$ ($a$ and $z$ in single's case), so \[G = \{ \{a_1,...,a_I\} \times \{o_1,...,o_J\}\}\]

Let $i$ refers to assets position and $j$ refers to values of everything else. Finally, indices $q$ refer to quadrature nodes, and $w_q$ to quadrature weights that add up to 1.

\subsection{Single Agents}
Consider females, males are handled analogously. Value function \ref{single-fem} can be written as
\[V^{f,s}_t(a,z) = \max\limits_{c} \left\{ u(c) + \beta \mathcal{V}^{f,s}_{t+1}(s,z)\right\}, \ \ c + s = M(a,z),\]
Here  $\mathcal{V}^{f,s}_{t+1}$ represents expectation of tomorrow's individual value function, accounting for all possible transitions. Argument $s$ refers to savings made in the current period (it is not necessary equal to future assets as future assets change if the partner is met), and $z$ refers to current value of productivity. Approximating $\mathcal{V}_{t+1}(s,z)$ (i.e. integration) is the key challenge. Note that it does not depend on current assets, only on savings $s$ that define future assets position. $M(a,z) = R\cdot a + \exp(z + \text{Trend}_t)$ is the amount of money in the end of the period.

I treat the function $\mathcal{V}^{f,s}_{t+1}(s,z)$ as a collection of functions $\mathcal{V}^{f,s}_{mj,t+1}$ for each value of current productivity $z_j$ and savings $s_m$. I generate quadrature nodes that consist of shock to individual productivity $\epsilon^{z,f}$, shocks to partner's assets and productivity positions
$(\epsilon^{a,p},\epsilon^{z,p})$ and initial couple's surplus $\psi$. I denote these nodes as $x_q = (\epsilon^{z,f}_q,\epsilon^{a,p},\epsilon^{z,p},\psi_q)$, in total monomial rule results in $Q = [...]$ nodes. Weight $w_q$ corresponds to each node. Given savings $s$ and grid point in $z_j$ I generate partner's and couple's characteristics. 
\[a^p_{mq} = s_m\cdot \exp(\epsilon^{a,p}_q), \ \ a^c_{mq} = s_m\cdot [1 + \exp(\epsilon^{a,p}_q)], \ \ z^m_{jq} = z^f_j \exp(\epsilon^{z,p}_q),\]
(in some sense in this setup by saving more individuals have chances of getting more wealthy partners).  Also, individual next period's productivity is generated according to $z^f_{jq} = z^f\cdot \exp(\epsilon^{z,f}_q)$. Note that $z^f_j$ refers to the current value and $z^f_{jq}$ refers to the next period's values;

This allows to generate potential couple's characteristics $(a^c_{mq},\psi_q,z^{f}_{jq},z^{m}_{jq})$, and combining it with value functions for the next period we can define:
\[\Theta^{np}_{mjq,t+1} = \left\{ \tilde\theta : V^{f,c}_{t+1}(a^c_{mjq}, \tilde\theta, \psi_q,z^{f}_{jq},z^{m}_{jq}) \geq V^{f,s}(s_m,z^f_{jq}), \ V^{m,c}_{t+1}(a^c_{mjq}, \tilde\theta, \psi_q,z^{f}_{jq},z^{m}_{jq}) \geq V^{m,s}(a^p_{mq},z^m_{jq}) \right\},\]
if $\Theta^{np}_{mjq,t+1}$ is empty\footnote{I figure this out by defining a uniformly spaced grid for $\theta \in [0.05,0.95]$. Because of Smolyak grid, this can be done very efficiently as $V^{f,c}_{t+1}(a^c_{mjq}, \tilde\theta, \psi_q,z^{f}_{jq},z^{m}_{jq})$ are just a polynomial in $\tilde{\theta}$.}
, $m^{np}_{mjq,t+1} = 0$ (so no marriage happens at node $jq$ if agents made savings of $s_m$). If the set is not empty, then $\theta_{mjq}$ is determined by Nash Bargaining \ref{nbs} and $m^{np}_{mjq,t+1} = 1$. Therefore at node $(s_m,z_j)$ future value function in case the partner is met and pregnancy did not happen can be approximated as
\[\E_t  \big\{ m^{np} \cdot V^{f,c}_{t+1}(\Omega^c) + (1-m^{np})V^{f,s}_t(\omega')\big\} \approx \mathcal{E}^{np}(s_m,z_j),\]
where
\[\mathcal{E}^{np}(s_m,z_j) = \sum\limits_{q=1}^{Q} w_q \cdot \left\{ m^{np}_{mjq,t+1}\cdot V^{f,c}_{t+1}(a^c_{mq},\theta_{mjq},\psi_q,z^{f\prime}_{jq},z^m_{jq}) + (1-m^{np}_{mjq,t+1})\cdot V^{f,s}_{t+1}(s_m,z_j)\right\}.\]
In exactly the same manner we define the set in the case the pregnancy shock did happen:
\[\Theta^{p}_{mjq,t+1} = \left\{ \tilde\theta : V^{f,ck}_{t+1}(a^c_{mjq}, \tilde\theta, \psi_q,z^{f}_j,z^{m}_{jq}) \geq V^{f,s?}(s_m,z^f_{j}), \ V^{m,ck}_{t+1}(a^c_{mjq}, \tilde\theta, \psi_q,z^{f}_{jq},z^{m}_{jq}) \geq V^{m,s}(a^p_{mq},z^m_{jq}) \right\},\]
where function $V^{f,s?} = p^{\text{out}}[V^{f,s} + S] + (1-p^{\text{out}})V^{f,sky}_t$ captures possible immediate transition out of being single mother. In the same way 
\begin{align*}\mathcal{E}^{p}(s_m,z_j) = \sum\limits_{q=1}^{Q} w_q \cdot  \Big\{ m^{p}_{mjq,t+1}\cdot V^{f,cky}_{t+1}(a^c_{mq},\theta_{mjq},\psi_q,z^f_j,z^m_{jq}) + (1-m^{p}_{mjq,t+1})\cdot V^{f,s?}(s_m,z_j) \Big\},
\end{align*}
lastly, the value of not meeting any partner is simply
\begin{align*}\mathcal{E}^{0}(s_m,z_j) = \sum\limits_{q=1}^{Q} w_q \cdot  V^{f,s}_{t+1}(s_m,z^f_{jq}),
\end{align*}
finally, the whole function $\mathcal{V}^{f,s}_{t+1}(s_m,z_j)$ can be approximated by weighted average
\[\mathcal{V}^{f,s}_{t+1}(s_m,z_j) \approx (1-p^{\text{meet}}_t) \cdot \mathcal{E}^{0}(s_m,z_j) +  p^{\text{meet}}_t(1-p^{\text{preg}}_t)  \cdot \mathcal{E}^{np}(s_m,z_j) 
+  p^{\text{meet}}_t p^{\text{preg}}_t  \cdot \mathcal{E}^{p}(s_m,z_j).\]

After that I use linear interpolation in $s$ dimension to approximate $\mathcal{V}^{f,s}_{t+1}(s_m,z_j)$ for arbitrary $s$. Therefore I can get approximation of $\mathcal{V}^{f,s}_{t+1}(s,z_j)$. As a result, value function in each asset point $a_i$, $z_j$ is obtained by numerically solving
\[V^{f,s}_t(a_i,z_j) = \max\limits_{s \in [\underline{A},\overline{A}]} \left\{ u(M(a_i,z_j) - s) + \beta \mathcal{V}^{f,s}_{t+1}(s,z_j)\right\}.\]
I take $\underline{A}$ and $\overline{A}$ the same as upper and lover bound for individual's assets grid (this is important as approximation may be crude). I use Matlab's \textbf{fminbnd} to solve for the maximum point, although I also test that results are very similar when I just use discrete grid $s \in \mathcal{S} = M(a_i,z_j) \cdot\{0,...,0.99\}$ with some spacing, though this variation is much slower.

Choice of solver can generically be an issue as because of concerns described above function $\mathcal{V}^{f,s}_{t+1}(s_m,z_j)$ may be non-smooth or even discontinuous.

\subsection{Couples}
I consider couples that do not have kids yet as they have richer set of possible transitions.  Following [...] I split household's decision problem on intrahousehold, intratemporal and intertemporal levels, where first two can be solved analytically. For intrahousehold, I define
\[U(c,\theta) = \max\limits_{c^f,c^m} \left\{\theta u(c^f)  + (1-\theta)u(c^m) \right\}, \ \ \text{s.t.} \ \ \ c = \left[ (c^f)^{1+\rho_c} + (c^m)^{1+\rho_c}\right]^{\frac1{1+\rho_c}},\]
and given CES assumption this function has simple analytical form.

Intratemporal dimension is trivial, as total consumption is the only intratemporal choice.

For intertemporal I can write the problem \ref{vf-c} as
\[V^{c}_{t+1}(a,\theta,\psi,z^f,z^m) = \max\limits_{c} \left\{ U(c,\theta) + \beta \mathcal{V}_{t+1}(s,\theta,\psi,z^f,z^m) \right\},\]
where $\mathcal{V}_{t+1}(s,\theta,\psi,z^f,z^m)$ consists of two terms: what happens if pregnancy shock arrives and what if it does not.

If the shock arrives, in the next period couple is in state $cky$, and if it decides to divorce female becomes a single mother (and may quit this state immediately), and male becomes just a single male. I approximate $\mathcal{V}_{t+1}(s,\theta,\psi,z^f,z^m)$ here. 

I write current state as combination of $(a_i,O_j)$. Function $\mathcal{V}_{t+1}(s,\theta,\psi,z^f,z^m)$ depends on current state $O= (\theta,\psi,z^f,z^m)$ and future savings $s$.  Integration nodes contains $x_q = (\epsilon^{z,f}_q,\epsilon^{z,m}_q,\epsilon^{\psi}_q)$. I take grid with respect to current states, so $O_j= (\theta_j,\psi_j,z^f_j,z^m_j) \in O$ and future savings $s \in A$, so I again approximate expected future value function at each combination of $s_m$ and $O_j$, so I compute values $\mathcal{V}_{mj,t+1} = \mathcal{V}_{t+1}(s_m,\theta_j,\psi_j,z^f_j,z^m_j)$. Given each $j$, I can compute next period values of $(\psi_{jq},z^f_{jq},z^m_{jq})$ by adding shocks given by quadrature notes to the current point. I also can compute divorce options: $\omega^{df}_{mjq} = ( (1-\kappa)\cdot 0.5 \cdot s_m, \ z^f_{mjq})$ and $\omega^{dm}_{mjq} = ( (1-\kappa)\cdot 0.5 \cdot s_m, \ z^m_{mjq})$. This allows to define for each quadrature node a renegotiation set:
\[\Omega^{p}_{mjq,t+1} = \left\{ \tilde\theta \ : \ V^{f,cky}_{t+1}(s_m,\tilde\theta,\psi_{jq},z^f_{jq},z^m_{jq}) \geq V^{f,s?}_{t+1}(\omega^{df}_{mjq}), \ \ V^{m,cky}_{t+1}(s_m,\tilde\theta,\psi_{jq},z^f_{jq},z^m_{jq}) \geq V^{m,s}_{t+1}(\omega^{dm}_{mjq}) \right\},\]
and then if $\Omega^{p}_{mjq,t+1}$ is not empty, I pick from it next period's $\theta_{mjq}$ closest to initial $\theta_j$ and set $d^{p}_{mjq} = 0$, otherwise divorce happens and $d^{p}_{mjq} = 0$.

Therefore expected future value if pregnancy shock arrives is 
{\small
\[\mathcal{E}^{p}(s_m,O_j) = \sum\limits_{q=1}^{Q} w_q \cdot\left\{ d^p_{mjq}\cdot V_{t+1}^{cky}(s_m,\theta_{mjq},\psi_{jq},z^f_{jq},z^m_{jq}) + (1-d^p_{mjq}) [\theta_{j}\cdot V^{f,sky}(\omega^{df}_{mjq}) + (1-\theta_j)V^{m,s}(\omega^{dm}_{mjq})] \right\} \]
}
If pregnancy shock does not arrive, then renegotiation should account for discrete decision of whether to have a kids that happens afterwards, and this decision may depend on $\theta$. Namely, I define
\[k_{mjq}(\tilde\theta) = \I\left[ V^{cky}_{t+1}(s_m,\tilde\theta,\psi_{jq},z^f_{jq},z^m_{jq}) \geq V^{c}_{t+1}(s_m,\tilde\theta,\psi_{jq},z^f_{jq},z^m_{jq})\right],\]
then I define individual value accounting for this choice:
\[V^{f,c?}_{t+1}(s_m,\tilde\theta,\psi_{jq},z^f_{jq},z^m_{jq}) = k_{mjq}(\tilde\theta)\cdot V^{f,cky}_{t+1}(s_m,\tilde\theta,\psi_{jq},z^f_{jq},z^m_{jq}) + (1-k_{mjq}(\tilde\theta))\cdot V^{f,c}_{t+1}(s_m,\tilde\theta,\psi_{jq},z^f_{jq},z^m_{jq}),\]
and the renegotiation set is defined as
\[\Omega^{np}_{mjq,t+1} = \left\{ \tilde\theta \ : \ V^{f,c?}_{t+1}(s_m,\tilde\theta,\psi_{jq},z^f_{jq},z^m_{jq}) \geq V^{f,s}_{t+1}(\omega^{df}_{mjq}), \ \ V^{m,c?}_{t+1}(s_m,\tilde\theta,\psi_{jq},z^f_{jq},z^m_{jq}) \geq V^{m,s}_{t+1}(\omega^{dm}_{mjq}) \right\},\]
and we define $\theta_{mjq}$ and $d^{np}_{mjq}$ analogously.

Therefore we define expected future value if pregnancy shock does not arrive to be
{\small
\[\mathcal{E}^{np}(s_m,O_j) = \sum\limits_{q=1}^{Q} w_q \cdot\left\{ d^{np}_{mjq}\cdot V_{t+1}^{c?}(s_m,\theta_{mjq},\psi_{jq},z^f_{jq},z^m_{jq}) + (1-d^{np}_{mjq}) [\theta_{j}\cdot V^{f,s}(\omega^{df}_{mjq}) + (1-\theta_j)V^{m,s}(\omega^{dm}_{mjq})] \right\} \]
}
so we can finally define value of $\mathcal{V}$ at points $s_m$:
\[\mathcal{V}^{c}_{t+1}(s_m,O_j) = p_t^{\text{preg}}\cdot \mathcal{E}^{p}(s_m,O_j) + (1-p_t^{\text{preg}})\cdot \mathcal{E}^{np}(s_m,O_j),\]
and then values at arbitrary points $s$ are obtained by linear interpolation. Therefore the ultimate problem is
\[V^{c}_t(a_i,O_j) = \max\limits_{s \in [\underline{A},\overline{A}]} \left\{ U(M(a_i,O_j) - s,\theta_j) + \beta \mathcal{V}^{c}_{t+1}(s,O_j)\right\}.\]


\subsection{Couples With Kids} 
Namely, for intrahousehold allocation of consumption I define
\[U(c,q,\theta) = \max\limits_{c^f,c^m} \left\{\theta u(c^f,q)  + (1-\theta)u(c^m,q) \right\}, \ \ \text{s.t.} \ \ \ c = \left[ (c^f)^{1+\rho_c} + (c^m)^{1+\rho_c}\right]^{\frac1{1+\rho_c}},\]
and it still has simple tractable form.

For intratemporal level I have to solve
\[U(m,W^f,\xi,\theta) = \max\limits_{c,q,l_f,x} U(c,q,\theta), \ \ \ \text{s.t.} \ \ \ c + x + l_f\cdot W^f = m, \ \ q =[\mu_\xi\cdot l_f^{\theta_q} + (1-\mu_\xi)\cdot x^{\theta_q}]^{\frac1{\theta_q}}. \]
this function can also be derived analytically given my assumption on functional forms.

After aggregation, the intertemporal problem looks similar to the one for singles:
\[V^{cky}(a,\theta,\psi,z^f,z^m) = \max\limits_{m} \left\{ U(m,W^f_t(z^f),\xi,\theta) + \beta \mathcal{V}^{ck}_{t+1}(s,\theta,\psi,z^f,z^m,\xi)\right\}, \ \ \text{s.t.} \ \ m + s = M(a,z^f,z^m),\]
where definition of $\mathcal{V}$ is analogous. % notes on solution process

\end{document}
