\documentclass[aspectratio=169]{beamer}
\usepackage[english]{babel}
\usepackage{amsmath,amsfonts}
\usepackage{multicol}

\usepackage{IEEEtrantools}
\usepackage{multirow}
% beamer setup

%\usecolortheme{dove}

\setbeamertemplate{navigation symbols}{}
\setbeamertemplate{itemize items}[ball]

\usepackage{tikz}
\usetikzlibrary{shapes,arrows}
\usetikzlibrary{positioning}
\tikzstyle{block} = [rectangle, draw, rounded corners]
\tikzstyle{line} = [draw, -latex']

\DeclareMathOperator*{\argmin}{argmin}
\DeclareMathOperator*{\argmax}{argmax}
\DeclareMathOperator{\E}{\mathbb{E}}
\DeclareMathOperator{\I}{\mathbb{I}}

\AtBeginSection[]{
  \begin{frame}[plain]
  \addtocounter{framenumber}{-1}
  \vfill
  \centering
  %\begin{beamercolorbox}[sep=8pt,center,shadow=true,rounded=true]{title}
    %\usebeamerfont{title}
    \Huge{\insertsectionhead\par}%
  %\end{beamercolorbox}
  \vfill
  \end{frame}
}


\newcommand{\backupbegin}{
   \newcounter{framenumberappendix}
   \setcounter{framenumberappendix}{\value{framenumber}}
}
\newcommand{\backupend}{
   \addtocounter{framenumberappendix}{-\value{framenumber}}
   \addtocounter{framenumber}{\value{framenumberappendix}} 
}




\title{The Economics of Shotgun Marriage}
\subtitle{and Household Bargaining}
\author{Egor Kozlov}

\institute{
  Department of Economics\\
  Northwestern University}
  
  
\setbeamertemplate{footline}[frame number]

  
%  \usepackage{pgf}
%\logo{\pgfputat{\pgfxy(0,0)}{\pgfbox[right,base]{\footnotesize{\insertframenumber\,/\,\inserttotalframenumber}}}}
%\newcommand{\nologo}{\setbeamertemplate{logo}{}}

\let\olditem\item
\renewcommand{\item}{%
\olditem\vspace{\fill}} 

\usepackage{xcolor}

\begin{document}



\begin{frame}[plain]
\addtocounter{framenumber}{-1}
\date{\scriptsize}
\titlepage
\end{frame}


\begin{frame}
\frametitle{Kids First or Marriage First}
\begin{itemize}
\item Around 1/4 of childbearing marriages in the US are \textbf{kids-first}:
\begin{itemize}
\item they have their first child \textit{before or in the year} of their marriage
\end{itemize}
\item These marriages have much higher divorce rates but are still widespread
\item What economic reasons make people do this? 
\item Unplanned pregnancies are a potential cause
\end{itemize}
\end{frame}

\begin{frame}
\frametitle{``America's strongest anti-poverty weapon''?}
\begin{itemize}
\item Marriage is a good thing (statistically)
%\item Conservatives blame millennials for not following ``the sequence''
\item Is making people marry more and stay married longer welfare improving?
\item The answer depends on understanding the economic \emph{value} of marriage
\item Simple competitive market setup says no (but may not fit here)
\end{itemize}
\end{frame}

\begin{frame}
\frametitle{Questions}
\begin{itemize}
\item Narrow: \begin{enumerate}
\item Why marriage / childbearing timing matters for divorce? Is it a causal relation?
\item How policies shape the creation and dissolution of the marriages? Should they?
\end{enumerate}
\item Broad:
\begin{enumerate}
\item How marriage works? What makes people enter ``risky'' marriages?
\end{enumerate}
\end{itemize}
\end{frame}


\begin{frame}
\frametitle{What I Do}
\begin{enumerate}
\item Describe empirically the behavior of kids-first and marriage-first couples
\item Use this empirical evidence to estimate a structural lifecycle model
\begin{itemize}
\item Singles meet potential partners and sometimes get unplanned pregnancies
\item Agents are heterogeneous in wages and savings, marriages differ in quality
\item People marry endogenously and cannot commit not to divorce
\end{itemize}
\item Use the model to talk about causality and policy
\begin{itemize}
\item Endogenous response to unplanned pregnancy explains the data well
\end{itemize}
\end{enumerate}
\end{frame}


\begin{frame}
\frametitle{Literature}
\begin{itemize}
\item Policy literature:
\begin{itemize}
\item Empirical: {\small \color{gray}Alessina, Guiliano, 2005;  Tannenbaum, 2020; Rossin-Slater, 2017}
\item Structural: {\small  \color{gray}Forester, 2020; Brown, Flinn, 2011; Kennes, Knowles, 2020}
\item I am the first to consider creation and dissolution of shotgun marriages structurally
\end{itemize}
\item Lifecycle literature:
\begin{itemize}
\item Dynamic limited commitment for marriage and divorce: {\small  \color{gray}Mazzocco, 2007; Voena, 2015; Low et al., 2018; Shephard, 2019; Blasutto, 2020}
\item Fertility during lifecycle: {\small  \color{gray}Sommer, 2015, Ejrnæs, Jørgensen, 2018}
\item I am the first to estimate a marriage/divorce model with fertility choice
\end{itemize}
\item Empirics on shotgun marriages (demography and sociology):
\begin{itemize}
\item {\small  \color{gray}Gibson-Davis et al (2016), Guzzo (2009),  Nuevo-Chiquero (2014)}
\item I can recover the causal content of these numbers
\end{itemize}
\end{itemize}
\end{frame}


\begin{frame}
\frametitle{Statistical Pattern: Kids-First $\Rightarrow$ Divorce More}
\begin{center}
\includegraphics[scale=0.5]{../pics/div_5y_by_dt.pdf}
\end{center}
\begin{itemize}
\item Controlling for many observable dimensions do not mitigate the difference
\item In subgroups, the pattern is sharper for college graduates
\item No exogenous variation, so the model is needed
\end{itemize}
\end{frame}


\begin{frame}
\frametitle{Conclusions}
\begin{enumerate}
\item Value of the marriage:
\begin{itemize}
\item Causality: kids-first marriage increases chances of future divorce by $\frac13$ to $\frac12$
\item Why do kids-first divorce more? Imperfect control over fertility, social stigma, disutility from stepchildren are the leading causes.
\item Selection: couples choose to have kids only if their risk of divorce is low enough
\item \textit{Risks} of unplanned pregnancies shape people's behavior quite a lot
\end{itemize}
\item Policy insights:
\begin{itemize}
\item Child support is inefficient in regulating marriage decisions for kids-first
\item Child support creates more single mothers but is still welfare improving
\item Pushing people towards marriage generally works, but only temporary
\end{itemize}
\end{enumerate}
\end{frame}


\begin{frame}
\frametitle{Key Takeaways}
\begin{itemize}
\item Kids-first marriages are adult analogues of teenage pregnancies
\item Reaction to unplanned pregnancies rationalizes the divorce pattern
\item Managing this reaction is difficult, but its causes can be controlled
\end{itemize}
\end{frame}

\end{document}


